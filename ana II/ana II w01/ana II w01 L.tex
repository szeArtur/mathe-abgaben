\documentclass{article}
\usepackage{graphicx, amsmath, amsthm, amssymb, mathtools, enumerate}
%\usepackage{virus}
\usepackage[a4paper, total={6in, 8in}]{geometry}
\usepackage[T1]{fontenc}
\usepackage[ngerman]{babel}


\title{Lineare Algebra II (LA) Übungsblatt 1}
\author{Erik Achilles, Alexandra Dittmar, Artur Szeczinowski}

\newcommand{\NN}{\mathbb{N}}
\newcommand{\ZZ}{\mathbb{Z}}
\newcommand{\QQ}{\mathbb{Q}}
\newcommand{\RR}{\mathbb{R}}
\newcommand{\FF}{\mathbb{F}}
\newcommand{\CC}{\mathbb{C}}

\newcommand{\imp}{\mathbb{\Rightarrow}}
\newcommand{\equ}{\mathbb{\Leftrightarrow}}
\newcommand{\eq}{\mathbb{\quad = \quad}}

\DeclareMathOperator{\RRe}{Re}
\DeclareMathOperator{\IIm}{Im}
\DeclareMathOperator{\Mat}{Mat}
\DeclareMathOperator{\im}{im}
\DeclareMathOperator{\rg}{rg}
\DeclareMathOperator{\LH}{L}
\DeclareMathOperator{\Bas}{Bas}
\DeclareMathOperator{\Kern}{ker}
\DeclareMathOperator{\Abb}{Abb}
\DeclareMathOperator{\Fin}{Fin}
\DeclareMathOperator{\Konv}{Konv}
\DeclareMathOperator{\Poly}{Poly}
\DeclareMathOperator{\arcosh}{arcosh}

\newcommand{\limn}{\lim_{n \rightarrow \infty}}
\newcommand{\toInf}[1]{\overset{#1 \rightarrow \infty}{\longrightarrow}}

\newcommand{\movs}[2]{\overset{\text{\tiny $#1$}}{\quad #2 \quad}}
\newcommand{\tovs}[2]{\overset{\text{\tiny (#1)}}{\quad #2 \quad}}
\newcommand{\vect}[1]{\begin{pmatrix*}[c] #1 \end{pmatrix*}}
\newcommand{\sect}[1]{\begin{psmallmatrix*}[c] #1 \end{psmallmatrix*}}
\newcommand{\legs}[2]{\left(\begin{array}{#1}#2\end{array}\right)}

\setlength{\parindent}{0pt}


\begin{document}

\maketitle
\vfill

\section*{Aufgabe 1.2}

Sei die Funktion

\[
\cosh : \RR \to \RR ,
x \mapsto \frac{e^x + e^{ - x}}{2}.
\]

\subsection*{a)}

Die Funktion 
$ \cosh $
ist differenzierbar, da
$ e^x, e^{ - x} $
und
$ 2 $
als Exponentialfunktionen bzw. konstante Funktionen differenzierbar sind. Die Differenzierbarkeit von 
$ \cosh $
folgt dann aus Summen - und Quotientenregel.
Wir bestimmen 
$ \sinh := \cosh' $:

\[
\begin{aligned}
    \sinh(x) = \cosh'(x)
    =
    \left(\frac{e^x + e^{ - x}}{2}\right)'
    & =
    \left( \frac{(e^x + e^{- x})' \cdot 2 -
    (e^x + e^{ - x}) \cdot 2'}{2^2} \right)
    \\ &=
    \left( \frac{(e^x - e^{- x}) \cdot 2 -
    (e^x + e^{ - x}) \cdot 0}{2 \cdot 2} \right)
    \\ &=
    \frac{(e^x - e^{ - x})}{2}.
\end{aligned}
\]

\newpage

\subsection*{b)}

Wir untersuchen 
$ \cosh $
auf lokale Extremstellen.
Wir bestimmen Nullstellen der ersten Ableitung:

\[
\begin{aligned}
    &\cosh'(x) = 0
    \\ \Leftrightarrow \qquad &
    \frac{(e^x - e^{ - x})}{2} = 0
    \\ \Leftrightarrow \qquad &
    e^x - e^{ - x} = 0
    \\ \Leftrightarrow \qquad &
    e^x = e^{ - x}
    \\ \Leftrightarrow \qquad &
    e^{2x} = e^{ - x} \cdot e^x = 1
    \\ \Leftrightarrow \qquad &
    2x = \ln(1) = 0
    \\ \Leftrightarrow \qquad &
    x = 0
\end{aligned}
\]

Also hat die Funktion ein lokales Extremum an der Stelle
$ x = 0 $.
Wir betrachten nun die zweite Ableitung an dieser Stelle:

\[
\cosh''(0)
=
\frac{e^0 + e^0}{2}
=
\frac{2}{2}
= 1 > 0.
\]

Also hat 
$ \cosh $
ein lokales Minimum an der Stelle
$ x = 0 $.

Wir untersuchen 
$ \cosh $
auf globale Extremstellen.
Für 
$ x > 0 $
gilt
$ e^x > 1, e^{ - x} < 1 $ das heißt

\[
\cosh'(x) = \frac{(e^x - e^{ - x})}{2}
> \frac{1 - 1}{2} = 0.
\]

Daraus folgt, dass 
$ \cosh(x) $ streng monoton wachsend ist.
Analog gilt für
$ x < 0 $,
dass
$ e^x < 1, e^{ - x} > 1 $ also

\[
\cosh'(x) = \frac{(e^x - e^{ - x})}{2}
< \frac{1 - 1}{2} = 0.
\]

Daraus folgt, dass 
$ \cosh(x) $ streng monoton fallend ist.
Somit hat die Funktion kein globales Maximum und ein globales Minimum an der Stelle
$ x = 0 $.

\newpage

\subsection*{c)}
Der Wertebereich der Funktion ist
$ \cosh(\RR) = [1, \infty) $.
\begin{proof}
    Aus dem globalen Minimum
    $ \cosh(0) = \frac{e^0 + e^0}{2} = 1 $
    und dem Monotonieverhalten folgern wir, dass
    $ \cosh(\RR) \subseteq [1,\infty] $.
    Wir zeigen ''$\supseteq$''.
    Sei ein beliebiges
    $ 1 \leq y < \infty $,
    dann gibt es ein
    $ X \in \RR $
    mit
    $$ \cosh(0) = 1 \leq y < \cosh(X). $$
    Da 
    $ \cosh $
    differenzierbar also insbesondere stetig ist, gibt es nach ZWS ein
    $ x \in [0,X] \subset \RR $
    mit
    $ \cosh(x) = y $.
    Somit gilt
    $ \cosh(\RR) = [1, \infty) $.
\end{proof}

\subsection*{d)}

Die Funktion
$ \cosh $
ist im Intervall
$ [0,\infty] $
stetig und streng monoton wachsend.
Also ist sie bijektiv  in diesem Intervall und hat eine Umkehrfunktion
$ \arcosh $.


\end{document}