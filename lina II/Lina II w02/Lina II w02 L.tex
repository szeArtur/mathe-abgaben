\documentclass{article}
\usepackage{graphicx, amsmath, amsthm, amssymb, mathtools, enumerate, bbm}
%\usepackage{virus}
\usepackage[a4paper, total={6in, 8in}]{geometry}
\usepackage[T1]{fontenc}
\usepackage[ngerman]{babel}


\title{Lineare Algebra II (LA) Übungsblatt 2}
\author{Erik Achilles, Alexandra Dittmar, Artur Szeczinowski}
\date{April 2025}




\setlength{\parindent}{0pt}


\newcommand{\NN}{\mathbb{N}}
\newcommand{\ZZ}{\mathbb{Z}}
\newcommand{\QQ}{\mathbb{Q}}
\newcommand{\RR}{\mathbb{R}}
\newcommand{\FF}{\mathbb{F}}
\newcommand{\CC}{\mathbb{C}}

\newcommand{\imp}{\mathbb{\Rightarrow}}
\newcommand{\equ}{\mathbb{\Leftrightarrow}}
\newcommand{\eq}{\mathbb{\quad = \quad}}

\DeclareMathOperator{\RRe}{Re}
\DeclareMathOperator{\IIm}{Im}
\DeclareMathOperator{\Mat}{Mat}
\DeclareMathOperator{\im}{im}
\DeclareMathOperator{\rg}{rg}
\DeclareMathOperator{\LH}{L}
\DeclareMathOperator{\Bas}{Bas}
\DeclareMathOperator{\Kern}{ker}
\DeclareMathOperator{\Abb}{Abb}
\DeclareMathOperator{\Fin}{Fin}
\DeclareMathOperator{\Konv}{Konv}
\DeclareMathOperator{\Poly}{Poly}

\newcommand{\limn}{\lim_{n \rightarrow \infty}}
\newcommand{\toInf}[1]{\overset{#1 \rightarrow \infty}{\longrightarrow}}

\newcommand{\movs}[2]{\overset{\text{\tiny $#1$}}{\quad #2 \quad}}
\newcommand{\tovs}[2]{\overset{\text{\tiny (#1)}}{\quad #2 \quad}}
\newcommand{\vect}[1]{\begin{pmatrix*}[c] #1 \end{pmatrix*}}
\newcommand{\sect}[1]{\begin{psmallmatrix*}[c] #1 \end{psmallmatrix*}}
\newcommand{\legs}[2]{\left(\begin{array}{#1}#2\end{array}\right)}


%% https://texblog.net/latex-archive/maths/amsmath-matrix/
\makeatletter
\renewcommand*\env@matrix[1][*\c@MaxMatrixCols c]{%
  \hskip -\arraycolsep
  \let\@ifnextchar\new@ifnextchar
  \array{#1}}
\makeatother




\begin{document}
%\maketitle
\section*{Aufgabe 2}

Zu
$A \in \Mat(m, K)$,
$C \in \Mat(n, K)$
definiere die Blockmatrix
$
B(A, C) := 
\begin{pmatrix*}
    A & 0 \\
    0 & C \\
\end{pmatrix*}
\in \Mat(m + n, K).
$
Wir zeigen:
$ \quad \det(B(A, C)) = \det(A) \cdot \det(C)$.


\subsection*{a)}
Sei
$\det(C) = 0$.
Nutze
\textit{(D6)}, \textit{(D7)}
und erhalte
$C'$
in oberer Dreiecksform mit
$\det(C')= \pm \det(C)= 0$. 

Aus
\textit{Satz 6.7}
folgt direkt

\[
\det(C') = 0
\tovs{6.7}{\equ}
\rg(C') = \tilde{\rg}(C') \neq n
\]

Da
$C'$
in oberer Dreiecksform steht
aber nicht vollen Rang hat,
muss mindestens die letzte Zeile null sein.
Also ist

\[
C'_{nn} = 0
\movs{}{\equ}
B(A, C')_{m+n,m+n} = 0
\tovs{D5}{\imp}
\det(B(A, C)) = \pm \det(B(A, C')) = 0.
\]


\subsection*{b)}
Sei nun
$\det(C) \neq 0$.
Wir definieren
$\widetilde{\det} : \Mat(m, K) \to K, \; A \mapsto \det(B(A, C)) \cdot \det(C)^{-1}$.

\bigbreak
(D1) $\widetilde{\det}$ ist zeilenlinear, denn für alle
$A \in \Mat(m, K)$
mit Zeilen
$\tilde{a}_1, \dots, \tilde{a}_m, \tilde{a}_i', \tilde{a}_i'' \in (K^m)^t$
und
$\lambda \in K$
gilt

\begin{align*}
    \widetilde{\det}\vect{\tilde{a}_1 \\ \vdots \\  \lambda(\tilde{a}_i'+\tilde{a}_i'') \\ \vdots \\ \tilde{a_m}}
    \eq 
    &\det\legs{cc}{\tilde{a}_1\\ \vdots \\  \lambda(\tilde{a}_i'+\tilde{a}_i'') & 0 \\ \vdots \\ \tilde{a_m} \\ 0 & C} \cdot \det(C)^{-1} \\
    \eq 
    &\lambda \cdot
    \det\legs{cc}{\tilde{a}_1\\ \vdots \\  \tilde{a}_i' & 0 \\ \vdots \\ \tilde{a_m} \\ 0 & C} 
    \cdot \det(C)^{-1}
    +
    \lambda \cdot
    \det\legs{cc}{\tilde{a}_1\\ \vdots \\  \tilde{a}_i'' & 0 \\ \vdots \\ \tilde{a_m} \\ 0 & C}
    \cdot \det(C)^{-1}\\
    \eq 
    &\lambda \cdot
    \widetilde{\det}\vect{\tilde{a}_1 \\ \vdots \\  \tilde{a}_i' \\ \vdots \\ \tilde{a_m}}+
    \lambda \cdot
    \widetilde{\det}\vect{\tilde{a}_1 \\ \vdots \\  \tilde{a}_i'' \\ \vdots \\ \tilde{a_m}}
\end{align*}

(D2) $\widetilde{\det}$ ist alternierend, denn falls
$\tilde{a}_i = \tilde{a}_j$,
$i \neq j$,
so ist auch
$\tilde{b}_i = (\tilde{a}_i \; 0) = (\tilde{a}_j \; 0) =\tilde{b}_j$
also
$\det(B(A,C)) = 0$
und somit
$\widetilde{\det}(A) = 0 \cdot \det(C)^{-1} = 0$.

\bigbreak
(D3) $\widetilde{\det}$ ist normiert, denn für
$B^{(m)} := B(\mathbbm{1}_m, C)$
streiche rekursiv die erste Zeile und Spalte.
Wir zeigen nun,
dass die Determinante bei dieser Striechung
erhalten bleibt. Dazu verwenden wir den
Spaltenentwicklungssatz von Laplace:

\begin{align*}
  \det(B^{(m)})
  &\eq
  \sum_{i=1}^{m+n} (-1)^{i+1} \cdot b_{i1} \cdot \det(B^{(m)\;\text{Str}}_{i1})\\
  &\eq
  (-1)^{1+1} \cdot b_{11} \cdot \det(B^{(m)\;\text{Str}}_{11}) + \sum_{i=2}^{m+n} (-1)^{i+1} \cdot b_{i1} \cdot \det(B^{(m)\;\text{Str}}_{i1}) \\
  &\eq
  1 \cdot 1 \cdot \det(B^{(m)\;\text{Str}}_{11}) + \sum_{i=2}^{m+n} (-1)^{i+1} \cdot 0 \cdot \det(B^{(m)\;\text{Str}}_{i1})\\
  &\eq
  \det(B^{(m)\;\text{Str}}_{11}) \quad = \quad \det(B^{(m-1)})
\end{align*}




Wir erhalten schließlich
$B^{(0)} = C$.
Also ist
$$\det(B^{(m)}) \cdot \det(C)^{-1}
\eq \det(B^{(0)}) \cdot \det(C)^{-1}
\eq \det(C) \cdot \det(C)^{-1}
\eq 1.$$
Da
$\widetilde{\det}$
\textit{(D1), (D2), (D3)}
erfüllt und die Determinantenabbildung eindeutig ist muss
$\widetilde{\det} = \det$ sein.
also ist
\begin{align*}
  & \det(A) = \widetilde{\det}(A) = \det(B(A, C)) \cdot \det(C)^{-1} \\
  \movs{}{\equ}
  & \det(A) \cdot \det(C) = \det(B(A, C))
\end{align*}



\end{document}
