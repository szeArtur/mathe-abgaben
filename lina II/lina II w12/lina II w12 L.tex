\documentclass{article}

\usepackage[utf8]{inputenc}
\usepackage{graphicx, amsmath, amsthm, amssymb, mathtools, enumerate, bbm}
\usepackage{stmaryrd}
\usepackage[a4paper, total={6in, 8in}]{geometry}
\usepackage[T1]{fontenc}
\usepackage[ngerman]{babel}


\title{Lineare Algebra II (LA) Übungsblatt 12}
\author{Erik Achilles, Alexandra Dittmar, Artur Szeczinowski}
\date{Juli 2025}



\setlength{\parindent}{0pt}



\newcommand{\NN}{\mathbb{N}}
\newcommand{\ZZ}{\mathbb{Z}}
\newcommand{\QQ}{\mathbb{Q}}
\newcommand{\RR}{\mathbb{R}}
\newcommand{\FF}{\mathbb{F}}
\newcommand{\CC}{\mathbb{C}}

\newcommand{\imp}{\mathbb{\Rightarrow}}
\newcommand{\equ}{\mathbb{\Leftrightarrow}}
\newcommand{\eq}{\mathbb{\quad = \quad}}

\newcommand{\limto}[2]{\lim_{#1 \rightarrow #2}}
\newcommand{\toinf}[1]{\overset{#1 \rightarrow \infty}{\longrightarrow}}

\DeclareMathOperator{\RRe}{Re}
\DeclareMathOperator{\IIm}{Im}
\DeclareMathOperator{\Mat}{Mat}
\DeclareMathOperator{\id}{id}
\DeclareMathOperator{\im}{im}
\DeclareMathOperator{\rg}{rg}
\DeclareMathOperator{\LH}{L}
\DeclareMathOperator{\Bas}{Bas}
\DeclareMathOperator{\Kern}{ker}
\DeclareMathOperator{\Abb}{Abb}
\DeclareMathOperator{\Fin}{Fin}
\DeclareMathOperator{\Konv}{Konv}
\DeclareMathOperator{\Poly}{Poly}
\DeclareMathOperator{\sign}{sign}
\DeclareMathOperator{\sgn}{sgn}
\DeclareMathOperator{\GL}{GL}
\DeclareMathOperator{\SL}{SL}
\DeclareMathOperator{\vol}{vol}
\DeclareMathOperator{\End}{End}
\DeclareMathOperator{\eigenraum}{Eig}



\begin{document}
\maketitle
%\newpage

\section*{Aufgabe 1}

 \subsection*{a)}
 \begin{proof}
    Für alle $v, w, z \in \RR^3$ gilt:

    \[
    \begin{aligned}
        v \times (w \times z)
        &=
        v \times
        \begin{pmatrix}
            w_2 z_3 - w_3 z_2 \\
            w_3 z_1 - w_1 z_3 \\
            w_1 z_2 - w_2 z_1
        \end{pmatrix}\\
        &=
        \begin{pmatrix}
            v_2 (w_1 z_2 - w_2 z_1) - v_3 (w_3 z_1 - w_1 z_3) \\
            v_3 (w_2 z_3 - w_3 z_2) - v_1 (w_1 z_2 - w_2 z_1) \\
            v_1 (w_3 z_1 - w_1 z_3) - v_2 (w_2 z_3 - w_3 z_2)
        \end{pmatrix} \\
        &=
        \begin{pmatrix}
            v_2 w_1 z_2 - v_2 w_2 z_1 - v_3 w_3 z_1 + v_3 w_1 z_3 \\
            v_3 w_2 z_3 - v_3 w_3 z_2 - v_1 w_1 z_2 + v_1 w_2 z_1 \\
            v_1 w_3 z_1 - v_1 w_1 z_3 - v_2 w_2 z_3 + v_2 w_3 z_2
        \end{pmatrix} \\
        &=
        \begin{pmatrix}
            w_1(v_2  z_2 + v_3  z_3) - z_1(v_2 w_2  + v_3 w_3 )  \\
            w_2(v_3  z_3 + v_1  z_1) - z_2(v_3 w_3  + v_1 w_1 )  \\
            w_3(v_1  z_1 + v_2  z_2) - z_3(v_1 w_1  + v_2 w_2 ) 
        \end{pmatrix} \\
        &=
        \begin{pmatrix}
            w_1(v_2  z_2 + v_3  z_3 + v_1  z_1 - v_1  z_1) - z_1(v_2 w_2  + v_3 w_3 + v_1  w_1 - v_1  w_1)  \\
            w_2(v_3  z_3 + v_1  z_1 + v_2  z_2 - v_2  z_2) - z_2(v_3 w_3  + v_1 w_1 + v_2  w_2 - v_2  w_2)  \\
            w_3(v_1  z_1 + v_2  z_2 + v_3  z_3 - v_3  z_3) - z_3(v_1 w_1  + v_2 w_2 + v_3  w_3 - v_3  w_3) 
        \end{pmatrix}  \\
        &=
        \begin{pmatrix}
            (w_1(v_2  z_2 + v_3  z_3 + v_1  z_1) - w_1 v_1  z_1) - (z_1(v_2 w_2  + v_3 w_3 + v_1  w_1) - z_1 v_1  w_1)  \\
            (w_2(v_3  z_3 + v_1  z_1 + v_2  z_2) - w_2 v_2  z_2) - (z_2(v_3 w_3  + v_1 w_1 + v_2  w_2) - z_2 v_2  w_2)  \\
            (w_3(v_1  z_1 + v_2  z_2 + v_3  z_3) - w_3 v_3  z_3) - (z_3(v_1 w_1  + v_2 w_2 + v_3  w_3) - z_3 v_3  w_3) 
        \end{pmatrix} \\
        &=
        \begin{pmatrix}
            w_1(v_2  z_2 + v_3  z_3 + v_1  z_1) - z_1(v_2 w_2  + v_3 w_3 + v_1  w_1) \\
            w_2(v_3  z_3 + v_1  z_1 + v_2  z_2) - z_2(v_3 w_3  + v_1 w_1 + v_2  w_2) \\
            w_3(v_1  z_1 + v_2  z_2 + v_3  z_3) - z_3(v_1 w_1  + v_2 w_2 + v_3  w_3) 
        \end{pmatrix} \\
        &=
        \langle v, z \rangle w - \langle v, w \rangle z
    \end{aligned}
    \]
    
 \end{proof}

 \newpage

 \subsection*{b)}
 Sei
 $\xi := (1,1,1) \in \RR^3$.
 Die Abbildung
 $\beta : \RR^3 \times \RR^3 \to \RR,
 \beta(v, w) := \langle v \times (\xi \times w), \xi \rangle$
 ist eine symmetrische Bilinearform.

 \begin{proof}
    Die Abbildung ist eine Verkettung von
    bilinearen Abbildung (Kreuzprodukt, Skalarprodukt)
    und ist daher ebenfalls bilinear.
    Wir zeigen nun Symmetrie.
    Dazu nutzen wir \textit{a)} und die Symmetrie des Skalarprodukts:

    \[
    \begin{aligned}
    \beta(v, w)
    =
    \langle v \times (\xi \times w), \xi \rangle
    \overset{a)}{=}
    \langle\langle v, w \rangle \xi - \langle v, \xi \rangle w, \xi\rangle
    =
    \langle v, w \rangle \langle \xi, \xi\rangle - \langle v, \xi \rangle \langle w, \xi \rangle
    \\
    =
    \langle w, v \rangle \langle \xi, \xi\rangle - \langle w, \xi \rangle \langle v, \xi \rangle   
    =
    \langle\langle w, v \rangle \xi - \langle w, \xi \rangle v, \xi\rangle
    \overset{a)}{=}
    \langle w \times (\xi \times v), \xi \rangle
    =
    \beta(w, v).
    \end{aligned}
    \]
 \end{proof}

 \newpage
 \section*{Aufgabe 2}
 Zu $M \in \Mat(n, K)$ definieren wir
 $M^{\pm} := \frac{1}{2}(M \pm M^t) \in \Mat(n, K)$.

 \subsection*{a)}
 Es gilt $M^{\pm} \in \Mat^{\pm}(n, K)$.

\begin{proof}
    \textit{Fall I} Gelte $M^{\pm} = \frac{1}{2}(M + M^t)$.
    Dann ist

    \[
    (M^{\pm})^t = (\frac{1}{2}(M + M^t))^t
    =
    \frac{1}{2}(M + M^t)^t
    =
    \frac{1}{2}(M^t + M)
    =
    \frac{1}{2}(M + M^t)
    =
    \frac{1}{2}(M + M^t)
    =
    M^{\pm}.
    \]

    \textit{Fall II} Gelte $M^{\pm} = \frac{1}{2}(M - M^t)$.
    Analog folgt $(M^{\pm})^t = M^{\pm}$.

    Also ist $M^{\pm} \in \{ A \in \Mat(n,K)| A^t = \pm A\}  = \Mat^{\pm}(n, K)$.
\end{proof}

\bigbreak\bigbreak

Es gilt $\Mat(n,K) = \Mat^+(n, K) \oplus \Mat^-(n, K)$.

\begin{proof}
    Da $\Mat^+(n, K), \Mat^-(n, K) \subset \Mat(n,K)$,
    ist $\Mat(n,K) \supseteq \Mat^+(n, K) \oplus \Mat^-(n, K)$.

    Wir zeigen ''$\subseteq$''.
    
    Sei $M \in \Mat(n,K)$ beliebig, dann gilt für passende
    $A \in \Mat^+(n,K), B \in \Mat^-(n,K)$:
    $M = A+B$, denn für alle
    $i,j \leq n \in \NN$ ist

    \[
    \begin{aligned}
        & M_{ij} = A_{ij} + B_{ij} \\
        \land\qquad
        & M_{ji} = A_{ji} + B_{ji} = A_{ij} - B_{ij}.
    \end{aligned}
    \]

    Dieser Ausdruck lässt sich als LGS schreiben:

    \[
    \begin{pmatrix}
        1 & 1 \\
        1 & -1
    \end{pmatrix}
    \begin{pmatrix}
        A_{ij} \\
        B_{ij}
    \end{pmatrix}
    =
    \begin{pmatrix}
        M_{ij} \\
        M_{ji}
    \end{pmatrix}.
    \]

    Da $\det\begin{psmallmatrix}
        1 & 1 \\
        1 & -1
    \end{psmallmatrix} = -2 \neq 0$
    hat das LGS für alle $i,j$ nur eine Lösung,
    d.h. $A$ und $B$ sind eindeutig bestimmt.
    Somit gilt $\Mat(n,K) \subseteq \Mat^+(n, K) \oplus \Mat^-(n, K)$.
\end{proof}

\end{document}