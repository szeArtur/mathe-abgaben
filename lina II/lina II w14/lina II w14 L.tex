\documentclass{article}

\usepackage[utf8]{inputenc}
\usepackage{graphicx, amsmath, amsthm, amssymb, mathtools, enumerate, bbm, graphicx}
\usepackage{stmaryrd}
\usepackage[a4paper, total={6in, 8in}]{geometry}
\usepackage[T1]{fontenc}
\usepackage[ngerman]{babel}


\title{Lineare Algebra II (LA) Übungsblatt 14}
\author{Erik Achilles, Alexandra Dittmar, Artur Szeczinowski}
\date{Juli 2025}



\setlength{\parindent}{0pt}



\newcommand{\NN}{\mathbb{N}}
\newcommand{\ZZ}{\mathbb{Z}}
\newcommand{\QQ}{\mathbb{Q}}
\newcommand{\RR}{\mathbb{R}}
\newcommand{\FF}{\mathbb{F}}
\newcommand{\CC}{\mathbb{C}}

\newcommand{\imp}{\mathbb{\Rightarrow}}
\newcommand{\equ}{\mathbb{\Leftrightarrow}}
\newcommand{\eq}{\mathbb{\quad = \quad}}

\newcommand{\limto}[2]{\lim_{#1 \rightarrow #2}}
\newcommand{\toinf}[1]{\overset{#1 \rightarrow \infty}{\longrightarrow}}

\DeclareMathOperator{\RRe}{Re}
\DeclareMathOperator{\IIm}{Im}
\DeclareMathOperator{\Mat}{Mat}
\DeclareMathOperator{\id}{id}
\DeclareMathOperator{\im}{im}
\DeclareMathOperator{\rg}{rg}
\DeclareMathOperator{\LH}{L}
\DeclareMathOperator{\Bas}{Bas}
\DeclareMathOperator{\Kern}{ker}
\DeclareMathOperator{\Abb}{Abb}
\DeclareMathOperator{\Fin}{Fin}
\DeclareMathOperator{\Konv}{Konv}
\DeclareMathOperator{\Poly}{Poly}
\DeclareMathOperator{\sign}{sign}
\DeclareMathOperator{\sgn}{sgn}
\DeclareMathOperator{\GL}{GL}
\DeclareMathOperator{\SL}{SL}
\DeclareMathOperator{\vol}{vol}
\DeclareMathOperator{\End}{End}
\DeclareMathOperator{\eigenraum}{Eig}


\begin{document}
\section*{Aufgabe 2}

Wir definieren
$g_j \in \Poly^3(\RR, \RR)$
durch

\[
g_0(x) : = 1,
\qquad
g_1(x) : = x - 1,
\qquad
g_2(x) : = x^2,
\qquad
g_3(x) : = x^3 - x^2.
\]

Weiter definieren wir die Basen
$B_0 : = (f_0, f_1, f_2, f_3)$
und
$B : = (g_0, g_1, g_2, g_3)$
von
$\Poly^3(\RR, \RR)$
und
$\Psi_B, \Psi_{B_0}$
die dazugehörigen Koordinatenisomorphismen.
Sei
$h > 0$.
Wir definieren die Abbildung

\[
\Delta_h : \Poly^3(\RR, \RR) \to \Poly^3(\RR, \RR),
\quad
(\Delta_h(f))(x) : = \frac{f(x + h) - f(x)}{h}.
\]

\subsection*{a)}

Die Abbildung
$\Delta_h$
ist linear.

\begin{proof}
    Seien
    $p_1, p_2 \in \Poly^3(\RR,\RR), \lambda \in \RR$
    beliebig.
    Dann gilt
\end{proof}


\end{document}