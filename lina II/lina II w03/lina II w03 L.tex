\documentclass{article}
\usepackage{graphicx, amsmath, amsthm, amssymb, mathtools, enumerate, bbm}
%\usepackage{virus}
\usepackage[a4paper, total={6in, 8in}]{geometry}
\usepackage[T1]{fontenc}
\usepackage[ngerman]{babel}


\title{Lineare Algebra II (LA) Übungsblatt 2}
\author{Erik Achilles, Alexandra Dittmar, Artur Szeczinowski}
\date{April 2025}




\setlength{\parindent}{0pt}


\newcommand{\NN}{\mathbb{N}}
\newcommand{\ZZ}{\mathbb{Z}}
\newcommand{\QQ}{\mathbb{Q}}
\newcommand{\RR}{\mathbb{R}}
\newcommand{\FF}{\mathbb{F}}
\newcommand{\CC}{\mathbb{C}}

\newcommand{\imp}{\mathbb{\Rightarrow}}
\newcommand{\equ}{\mathbb{\Leftrightarrow}}
\newcommand{\eq}{\mathbb{\quad = \quad}}

\DeclareMathOperator{\RRe}{Re}
\DeclareMathOperator{\IIm}{Im}
\DeclareMathOperator{\Mat}{Mat}
\DeclareMathOperator{\im}{im}
\DeclareMathOperator{\rg}{rg}
\DeclareMathOperator{\LH}{L}
\DeclareMathOperator{\Bas}{Bas}
\DeclareMathOperator{\Kern}{ker}
\DeclareMathOperator{\Abb}{Abb}
\DeclareMathOperator{\Fin}{Fin}
\DeclareMathOperator{\Konv}{Konv}
\DeclareMathOperator{\Poly}{Poly}
\DeclareMathOperator{\sign}{sign}

\newcommand{\limn}{\lim_{n \rightarrow \infty}}
\newcommand{\toInf}[1]{\overset{#1 \rightarrow \infty}{\longrightarrow}}

\newcommand{\movs}[2]{\overset{\text{\tiny $#1$}}{\quad #2 \quad}}
\newcommand{\tovs}[2]{\overset{\text{\tiny (#1)}}{\quad #2 \quad}}
\newcommand{\vect}[1]{\begin{pmatrix*}[c] #1 \end{pmatrix*}}
\newcommand{\sect}[1]{\begin{psmallmatrix*}[c] #1 \end{psmallmatrix*}}
\newcommand{\legs}[2]{\left(\begin{array}{#1}#2\end{array}\right)}


%% https://texblog.net/latex-archive/maths/amsmath-matrix/
\makeatletter
\renewcommand*\env@matrix[1][*\c@MaxMatrixCols c]{%
  \hskip -\arraycolsep
  \let\@ifnextchar\new@ifnextchar
  \array{#1}}
\makeatother




\begin{document}
%\maketitle
\section*{Aufgabe 2}

\subsection*{a)}
Wir betrachten die Permutation
\[
  \sigma := \begin{pmatrix}
    1 & 2 & 3 & 4 & 5 & 6 & 7 \\
    3 & 4 & 1 & 7 & 5 & 2 & 6
  \end{pmatrix} \in S_7.
\]
''$n=7$''
Es ist
$\sigma(7) = 6$
also ''Fall 2''.
Sei
$\tau_1$
die Transposition,
die $6$ und $7$ vertauscht:
\[
  \tau_1 := \begin{pmatrix}
    1 & 2 & 3 & 4 & 5 & 6 & 7 \\
    1 & 2 & 3 & 4 & 5 & 7 & 6
  \end{pmatrix}.
\]
Für
$(\tau_1 \circ \sigma)$
gilt nun ''Fall 1''.
Wir wenden das Verfahren also erneut auf den Rest
$\sigma'$ an, wobei
\[
  \sigma' := (\tau_1 \circ \sigma)|_{X_6} = \begin{pmatrix}
    1 & 2 & 3 & 4 & 5 & 6 \\
    3 & 4 & 1 & 6 & 5 & 2
  \end{pmatrix} \in S_6.
\]
''$n=6$''
Es ist
$\sigma'(6) = 2$
also ''Fall 2''.
Sei
$\tau_2'$
die Transposition,
die $2$ und $6$ vertauscht:
\[
  \tau_2' := \begin{pmatrix}
    1 & 2 & 3 & 4 & 5 & 6 \\
    1 & 6 & 3 & 4 & 5 & 2
  \end{pmatrix}.
\]
Für
$(\tau_2' \circ \sigma')$
gilt nun ''Fall 1''.
Wir wenden das Verfahren erneut auf den Rest
$\sigma''$ an, wobei
\[
  \sigma'' := (\tau_2' \circ \sigma')|_{X_5} = \begin{pmatrix}
    1 & 2 & 3 & 4 & 5 \\
    3 & 4 & 1 & 2 & 5
  \end{pmatrix}.
\]
''$n=5$'' Es ist
$\sigma''(5) = 5$.
Für
$\sigma''$
gilt daher ''Fall 1''.
Wir wenden das Verfahren erneut auf $\sigma'''$ an, wobei
\[
  \sigma''' := \sigma''|_{X_4} = \begin{pmatrix}
    1 & 2 & 3 & 4 \\
    3 & 4 & 1 & 2
  \end{pmatrix}.
\]
''$n=4$''
Es ist
$\sigma'''(4) = 2$
also ''Fall 2''.
Sei
$\tau_3'''$
die Transposition,
die $2$ und $4$ vertauscht:
\[
  \tau_3''' := \begin{pmatrix}
    1 & 2 & 3 & 4 \\
    1 & 4 & 3 & 2
  \end{pmatrix}.
\]
Für
$(\tau_3''' \circ \sigma''')$
gilt nun ''Fall 1''.
Wir wenden das Verfahren erneut auf den Rest
$\sigma''''$ an, wobei
\[
  \sigma'''' := (\tau_3''' \circ \sigma''')|_{X_3} = \begin{pmatrix}
    1 & 2 & 3 \\
    3 & 2 & 1
  \end{pmatrix}.
\]
''$n=3$''
Es ist
$\sigma''''(3) = 1$
also ''Fall 2''.
Sei
$\tau_4''''$
die Transposition,
die $1$ und $3$ vertauscht:
\[
  \tau_4'''' := \begin{pmatrix}
    1 & 2 & 3 \\
    3 & 2 & 1
  \end{pmatrix}.
\]
Für
$(\tau_4'''' \circ \sigma'''')$
gilt nun ''Fall 1''.
Wir wenden das Verfahren also erneut auf den Rest
$\sigma'''''$ an, wobei
\[
  \sigma''''' := (\tau_3'''' \circ \sigma'''')|_{X_2} = \begin{pmatrix}
    1 & 2 \\
    1 & 2
  \end{pmatrix}.
\]
''$n=2$'' Es ist
$\sigma'''''(2) = 2$.
Für
$\sigma'''''$
gilt daher ''Fall 1''.
Wir erhalten den Rest
$\sigma'''''' := \sigma'''''|_{X_1} = id \in S_1$
und haben so den ''Induktionsanfang'' erreicht.
\newpage
Wir definieren eine ''Ergänzungsabbildung''
\[
  \varphi : S_n \to S_7, \; \tau' \mapsto \tau
  \qquad \text{mit} \qquad
  \tau(k) :=
  \begin{cases*}
    \tau'(k) & \text{für $k \in X_n$} \\
    k        & \text{sonst}
  \end{cases*}.
\]
Nun können wir alle Transpositionen ergänzen:
\begin{align*}
  \tau_1 =                        & \begin{pmatrix}
                                      1 & 2 & 3 & 4 & 5 & 6 & 7 \\
                                      1 & 2 & 3 & 4 & 5 & 7 & 6
                                    \end{pmatrix}, \\
  \tau_2 := \varphi(\tau_2') =    & \begin{pmatrix}
                                      1 & 2 & 3 & 4 & 5 & 6 & 7 \\
                                      1 & 6 & 3 & 4 & 5 & 2 & 7
                                    \end{pmatrix}, \\
  \tau_3 := \varphi(\tau_3''') =  & \begin{pmatrix}
                                      1 & 2 & 3 & 4 & 5 & 6 & 7 \\
                                      1 & 4 & 3 & 2 & 5 & 6 & 7
                                    \end{pmatrix}, \\
  \tau_4 := \varphi(\tau_4'''') = & \begin{pmatrix}
                                      1 & 2 & 3 & 4 & 5 & 6 & 7 \\
                                      3 & 2 & 1 & 4 & 5 & 6 & 7
                                    \end{pmatrix}.
\end{align*}
Nun ist $\sigma = \tau_1 \circ \tau_2 \circ \tau_3 \circ \tau_4$.


\subsection*{b)}
Wir berechnen
$\epsilon(\sigma, 2, 6)$:
\[
  \epsilon(\sigma, 2, 6)
  =
  \sign\left(\frac{\sigma(6) - \sigma(2)}{6 - 2}\right)
  =
  \sign\left(\frac{2 - 4}{6 - 2}\right)
  =
  \sign\left(-\frac{1}{2}\right)
  = -1.
\]
Wir berechnen
$\epsilon(\sigma, 6, 2)$:
\[
  \epsilon(\sigma, 6, 2)
  =
  \sign\left(\frac{\sigma(2) - \sigma(1)}{2 - 6}\right)
  =
  \sign\left(\frac{4 - 2}{2 - 6}\right)
  =
  \sign\left(-\frac{1}{2}\right)
  = -1.
\]
Wir berechnen
$\epsilon(\sigma, 1, 7)$:
\[
  \epsilon(\sigma, 1, 7)
  =
  \sign\left(\frac{\sigma(7) - \sigma(1)}{7 - 1}\right)
  =
  \sign\left(\frac{6 - 3}{7 - 1}\right)
  =
  \sign\left(\frac{1}{2}\right)
  = 1
\]



\end{document}
