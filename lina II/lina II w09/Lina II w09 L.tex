\documentclass{article}
\usepackage{graphicx, amsmath, amsthm, amssymb, mathtools, enumerate, bbm}
\usepackage{stmaryrd}
\usepackage[a4paper, total={6in, 8in}]{geometry}
\usepackage[T1]{fontenc}
\usepackage[ngerman]{babel}


\title{Lineare Algebra II (LA) Übungsblatt 9}
\author{Erik Achilles, Alexandra Dittmar, Artur Szeczinowski}
\date{Mai 2025}




\setlength{\parindent}{0pt}

\newcommand{\NN}{\mathbb{N}}
\newcommand{\ZZ}{\mathbb{Z}}
\newcommand{\QQ}{\mathbb{Q}}
\newcommand{\RR}{\mathbb{R}}
\newcommand{\FF}{\mathbb{F}}
\newcommand{\CC}{\mathbb{C}}

\newcommand{\imp}{\mathbb{\Rightarrow}}
\newcommand{\equ}{\mathbb{\Leftrightarrow}}
\newcommand{\eq}{\mathbb{\quad = \quad}}

\newcommand{\limn}{\lim_{n \rightarrow \infty}}
\newcommand{\toInf}[1]{\overset{#1 \rightarrow \infty}{\longrightarrow}}

\newcommand{\movs}[2]{\overset{\text{\tiny $#1$}}{\quad #2 \quad}}
\newcommand{\tovs}[2]{\overset{\text{\tiny (#1)}}{\quad #2 \quad}}
\newcommand{\vect}[1]{\begin{pmatrix*}[c] #1 \end{pmatrix*}}
\newcommand{\sect}[1]{\begin{psmallmatrix*}[c] #1 \end{psmallmatrix*}}
\newcommand{\legs}[2]{\left(\begin{array}{#1}#2\end{array}\right)}

\DeclareMathOperator{\RRe}{Re}
\DeclareMathOperator{\IIm}{Im}
\DeclareMathOperator{\Mat}{Mat}
\DeclareMathOperator{\im}{im}
\DeclareMathOperator{\rg}{rg}
\DeclareMathOperator{\LH}{L}
\DeclareMathOperator{\Bas}{Bas}
\DeclareMathOperator{\Kern}{ker}
\DeclareMathOperator{\Abb}{Abb}
\DeclareMathOperator{\Fin}{Fin}
\DeclareMathOperator{\Konv}{Konv}
\DeclareMathOperator{\Poly}{Poly}
\DeclareMathOperator{\sign}{sign}
\DeclareMathOperator{\sgn}{sgn}
\DeclareMathOperator{\GL}{GL}
\DeclareMathOperator{\SL}{SL}
\DeclareMathOperator{\vol}{vol}
\DeclareMathOperator{\End}{End}
\DeclareMathOperator{\eigenraum}{Eig}

%% https://texblog.net/latex-archive/maths/amsmath-matrix/
\makeatletter
\renewcommand*\env@matrix[1][*\c@MaxMatrixCols c]{%
  \hskip -\arraycolsep
  \let\@ifnextchar\new@ifnextchar
  \array{#1}}
\makeatother




\begin{document}
%\maketitle
%\newpage

\section*{Aufgabe 1}

Wir betrachten die folgenden Matrizen aus
$\Mat(3, \RR)$:

\[
  A := \left(
  \begin{array}{ccc}
      -4 & -3 & -3 \\
      3  & 2  & 3  \\
      3  & 3  & 3  \\
    \end{array}
  \right)
  \qquad \text{und} \qquad
  B := \left(
  \begin{array}{ccc|ccc}
      1 & 2 & 3 & 0 & 0 \\
      0 & 1 & 2 & 0 & 0 \\
      0 & 0 & 1 & 0 & 0 \\ \hline
      0 & 0 & 0 & 4 & 0 \\
      0 & 0 & 0 & 5 & 4
    \end{array}
  \right)
\]

\subsection*{a)}

\subsubsection*{i)}

\subsubsection*{ii)}

\subsection*{b)}

\newpage

\section*{Aufgabe 2}

Sei $V$ ein $K$-Vektorraum,
$\lambda, \mu \in K$
sowie
$\varphi \in \End(V)$
und
$A \in \Mat(n, K)$.

\subsection*{a)}

Die Abbildung $\varphi$ ist genau dann injektiv,
wenn $0$ kein Eigenwert von $\varphi$ ist.

\begin{proof}

\end{proof}

\subsection*{b)}

Ist $\varphi$ bijektiv und $\lambda$ ein Eigenwert von $\varphi$,
so folgt
$\lambda \neq 0$
und $\lambda^{-1}$ ist Eigenwert von $\varphi^{-1}$.

\begin{proof}

\end{proof}

\subsection*{c)}

Gilt
$p \in K[X]$
und
$v \in \eigenraum(A, \lambda)$,
so folgt
$v \in \eigenraum(\tilde{p}(A), \tilde{p}(\lambda))$.

\begin{proof}

\end{proof}

\subsection*{d)}

Seien
$v_{\lambda}, v_{\mu} \in K^n$
Eigenvektoren von $A$ zu den Eigenwerten
$\lambda, \mu$.
Dann ist
$v_{\lambda} + v_{\mu}$
wieder ein Eigenvektor,
wenn --------------------

\begin{proof}

\end{proof}

\end{document}
