\documentclass{article}
\usepackage{graphicx, amsmath, amsthm, amssymb, mathtools, enumerate, bbm}
\usepackage{stmaryrd}
\usepackage[a4paper, total={6in, 8in}]{geometry}
\usepackage[T1]{fontenc}
\usepackage[ngerman]{babel}


\title{Lineare Algebra II (LA) Übungsblatt 9}
\author{Erik Achilles, Alexandra Dittmar, Artur Szeczinowski}
\date{Mai 2025}



\setlength{\parindent}{0pt}



\newcommand{\NN}{\mathbb{N}}
\newcommand{\ZZ}{\mathbb{Z}}
\newcommand{\QQ}{\mathbb{Q}}
\newcommand{\RR}{\mathbb{R}}
\newcommand{\FF}{\mathbb{F}}
\newcommand{\CC}{\mathbb{C}}

\newcommand{\imp}{\mathbb{\Rightarrow}}
\newcommand{\equ}{\mathbb{\Leftrightarrow}}
\newcommand{\eq}{\mathbb{\quad = \quad}}

\newcommand{\limto}[2]{\lim_{#1 \rightarrow #2}}
\newcommand{\toinf}[1]{\overset{#1 \rightarrow \infty}{\longrightarrow}}

\DeclareMathOperator{\RRe}{Re}
\DeclareMathOperator{\IIm}{Im}
\DeclareMathOperator{\Mat}{Mat}
\DeclareMathOperator{\id}{id}
\DeclareMathOperator{\im}{im}
\DeclareMathOperator{\rg}{rg}
\DeclareMathOperator{\LH}{L}
\DeclareMathOperator{\Bas}{Bas}
\DeclareMathOperator{\Kern}{ker}
\DeclareMathOperator{\Abb}{Abb}
\DeclareMathOperator{\Fin}{Fin}
\DeclareMathOperator{\Konv}{Konv}
\DeclareMathOperator{\Poly}{Poly}
\DeclareMathOperator{\sign}{sign}
\DeclareMathOperator{\sgn}{sgn}
\DeclareMathOperator{\GL}{GL}
\DeclareMathOperator{\SL}{SL}
\DeclareMathOperator{\vol}{vol}
\DeclareMathOperator{\End}{End}
\DeclareMathOperator{\eigenraum}{Eig}



\begin{document}
%\maketitle
%\newpage

\section*{Aufgabe 1}

Wir betrachten die folgenden Matrizen aus
$\Mat(3, \RR)$:

\[
  A := \left(
  \begin{array}{ccc}
      -4 & -3 & -3 \\
      3  & 2  & 3  \\
      3  & 3  & 3  \\
    \end{array}
  \right)
  \qquad \text{und} \qquad
  B := \left(
  \begin{array}{ccc|ccc}
      1 & 2 & 3 & 0 & 0 \\
      0 & 1 & 2 & 0 & 0 \\
      0 & 0 & 1 & 0 & 0 \\ \hline
      0 & 0 & 0 & 4 & 0 \\
      0 & 0 & 0 & 5 & 4
    \end{array}
  \right)
\]



\subsection*{a)}

\subsubsection*{i)}

\subsubsection*{ii)}

\subsection*{b)}

\newpage

\section*{Aufgabe 2}

Sei $V$ ein $K$-Vektorraum,
$\lambda, \mu \in K$
sowie
$\varphi \in \End(V)$
und
$A \in \Mat(n, K)$.

\subsection*{a)}

Die Abbildung $\varphi$ ist genau dann injektiv,
wenn $0$ kein Eigenwert von $\varphi$ ist.

\begin{proof}
Es gilt:

\[
\begin{aligned}
  &\text{$\varphi$ ist injektiv}\\
  \qquad\Leftrightarrow\qquad
  &\Kern(\varphi) = \Kern(\varphi - 0 \cdot \id_v) = \eigenraum(\varphi, 0) \neq 0 \\
  \qquad\Leftrightarrow\qquad
  &\text{$0$ ist kein Eigenwert von $v$}
\end{aligned}
\]
\end{proof}

\subsection*{b)}

Ist $\varphi$ bijektiv und $\lambda$ ein Eigenwert von $\varphi$,
so folgt
$\lambda \neq 0$
und $\lambda^{-1}$ ist Eigenwert von $\varphi^{-1}$.

\begin{proof}
  Sei $\lambda$ ein Eigenwert von $\varphi$
  und $v \in \eigenraum(\varphi, \lambda)$ d.h.
  $\varphi(v) = \lambda v$ und insbesondere $\varphi^{-1}(\lambda v) = v$.
  Daraus folgt

  \[
  \varphi(\lambda^{-1} v)
  = \lambda^{-1} \varphi(v)
  = \lambda^{-1} \varphi(\varphi^{-1}(\lambda v))
  = v.
  \]

  Das ist äquivalent zu $\varphi^{-1}(v) = \lambda^{-1} v$.
  Also ist $\lambda^{-1}$ Eigenwert von $\varphi^{-1}$.
\end{proof}

\subsection*{c)}

Gilt
$p \in K[X]$
und
$v \in \eigenraum(A, \lambda)$,
so folgt
$v \in \eigenraum(\tilde{p}(A), \tilde{p}(\lambda))$.

\begin{proof}
  Sei $p \in K[X]$
  und
  $v \in \eigenraum(A, \lambda) = \Kern(A - \lambda \mathbbm{1}_3)$,
  d.h.

  \[
  (A - \lambda \mathbbm{1}_3) \cdot v = 0
  \qquad\text{also auch}\qquad
  \tilde{p}((A - \lambda \mathbbm{1}_3) \cdot v) \;=\; \tilde{p}(0) \;=\; 0.
  \]

  Aufgrund von \textit{Lemma 7.25} folgt

  \[
  ((\tilde{p}(A) -\tilde{p}(\lambda) \mathbbm{1}_3) \cdot \tilde{p}(v))
  \;=\;
  \tilde{p}((A - \lambda \mathbbm{1}_3) \cdot v)
  \;=\;
  0.
  \]

  Also ist $\tilde{p}(v) \in \eigenraum(\tilde{p}(A), \tilde{p}(\lambda))$
  und somit auch $(A - \lambda \mathbbm{1}_3) \cdot v = 0$,
  da $v$ und $\tilde{p}(v)$ l.a. sind.
\end{proof}

\subsection*{d)}

Seien
$v_{\lambda}, v_{\mu} \in K^n$
Eigenvektoren von $A$ zu den Eigenwerten
$\lambda, \mu$.
Dann ist
$v_{\lambda} + v_{\mu}$
wieder ein Eigenvektor,
wenn --------------------

\begin{proof}

\end{proof}

\end{document}
