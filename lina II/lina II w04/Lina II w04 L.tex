\documentclass{article}
\usepackage{graphicx, amsmath, amsthm, amssymb, mathtools, enumerate, bbm}
%\usepackage{virus}
\usepackage[a4paper, total={6in, 8in}]{geometry}
\usepackage[T1]{fontenc}
\usepackage[ngerman]{babel}


\title{Lineare Algebra II (LA) Übungsblatt 4}
\author{Erik Achilles, Alexandra Dittmar, Artur Szeczinowski}
\date{Mai 2025}




\setlength{\parindent}{0pt}


\newcommand{\NN}{\mathbb{N}}
\newcommand{\ZZ}{\mathbb{Z}}
\newcommand{\QQ}{\mathbb{Q}}
\newcommand{\RR}{\mathbb{R}}
\newcommand{\FF}{\mathbb{F}}
\newcommand{\CC}{\mathbb{C}}

\newcommand{\imp}{\mathbb{\Rightarrow}}
\newcommand{\equ}{\mathbb{\Leftrightarrow}}
\newcommand{\eq}{\mathbb{\quad = \quad}}

\DeclareMathOperator{\RRe}{Re}
\DeclareMathOperator{\IIm}{Im}
\DeclareMathOperator{\Mat}{Mat}
\DeclareMathOperator{\im}{im}
\DeclareMathOperator{\rg}{rg}
\DeclareMathOperator{\LH}{L}
\DeclareMathOperator{\Bas}{Bas}
\DeclareMathOperator{\Kern}{ker}
\DeclareMathOperator{\Abb}{Abb}
\DeclareMathOperator{\Fin}{Fin}
\DeclareMathOperator{\Konv}{Konv}
\DeclareMathOperator{\Poly}{Poly}
\DeclareMathOperator{\sign}{sign}
\DeclareMathOperator{\sgn}{sgn}
\DeclareMathOperator{\GL}{GL}
\DeclareMathOperator{\SL}{SL}

\newcommand{\limn}{\lim_{n \rightarrow \infty}}
\newcommand{\toInf}[1]{\overset{#1 \rightarrow \infty}{\longrightarrow}}

\newcommand{\movs}[2]{\overset{\text{\tiny $#1$}}{\quad #2 \quad}}
\newcommand{\tovs}[2]{\overset{\text{\tiny (#1)}}{\quad #2 \quad}}
\newcommand{\vect}[1]{\begin{pmatrix*}[c] #1 \end{pmatrix*}}
\newcommand{\sect}[1]{\begin{psmallmatrix*}[c] #1 \end{psmallmatrix*}}
\newcommand{\legs}[2]{\left(\begin{array}{#1}#2\end{array}\right)}


%% https://texblog.net/latex-archive/maths/amsmath-matrix/
\makeatletter
\renewcommand*\env@matrix[1][*\c@MaxMatrixCols c]{%
  \hskip -\arraycolsep
  \let\@ifnextchar\new@ifnextchar
  \array{#1}}
\makeatother




\begin{document}
%\maketitle
%\newpage
\section*{Aufgabe 1}

\subsection*{a)}
Für eine beliebige Matrix
$A \in \Mat(3,\ZZ)$
gilt:

\[
\text{$A$ ist invertierbar und es gilt $A^{-1} \in \Mat(3,\ZZ)$}
\qquad\equ\qquad
\det(A) \in \{1,-1\}.
\]
\begin{proof}
  ''$\Rightarrow$''  
  Sei $A \in \Mat(3,\ZZ)$
  und gelte
  $A^{-1} \in \Mat(3,\ZZ)$.
  Da alle Matrixeinträge
  ganzzahlig sind, und
  nach Spaltenentwicklungssatz von Laplace
  sich die Determinante als Summe von
  Produkten der Einträge darstellen lässt, ist auch
  $\det(A), \det(A^{-1}) \in \ZZ$.
  Außerdem ist
  \[
  \det(A) \cdot \det(A^{-1})
  \eq
  \det(A \cdot A^{-1})
  \eq
  \det(\mathbbm{1}_3)
  \eq
  1.
  \]
  Also sind
  $\det(A)$ und $\det(A^{-1})$
  ganzzahlige Teiler der $1$ und somit
  $\det(A), \det(A^{-1}) \in \{1,-1\}$.

  \bigbreak

  ''$\Leftarrow$''
  Sei $A \in \Mat(3,\ZZ)$
  und gelte
  $\det(A) \in \{1,-1\}$.
  Nach \textit{Satz 6.7}
  ist $A$ invertierbar.
  Für die Inverse gilt zudem:
  \[
  A^{-1} = \frac{1}{\det(A)} \cdot B = \pm B
  \]
  wenn $B$ durch
  $b_{ij} := (-1)^{i+j} \cdot \det(A_{ji}^{\text{Str}})$
  gegeben ist. Somit ist jedes $a_{ij} = \pm b_{ij}$ durch ein Produkt
  ganzer Zahlen gegeben und somit auch ganzzahlig:
  $\det(A^{-1}) \in \Mat(3,\ZZ)$.
\end{proof}

\subsection*{b)}

$\SL(3,\ZZ) := \{A \in \Mat(3,\ZZ) | \det(A) = 1\}$
ist eine Untergruppe von
$\GL(3,\RR)$

\begin{proof}
  Wir zeigen die 3 Eigenschaften separat.

  \textbf{1.}
  Es gilt
  $\det(\mathbbm{1}_n) = 1$,
  also ist
  $\mathbbm{1}_n \in \SL(3,\ZZ)$.

  \textbf{2.}
  Seien
  $A,B \in \SL(3,\ZZ)$,
  so gilt
  $\det(A \cdot B) = \det(A) \cdot \det(B) = 1 \cdot 1 = 1$,
  also ist
  $A \cdot B \in \SL(3,\ZZ)$.

  \textbf{3.}
  Sei
  $A \in \SL(3,\ZZ)$,
  also
  $A \in \Mat(3,\ZZ)$ und $\det(A) = 1$.
  Wie in \textit{a)} gezeigt wurde, gibt es ein
  Inverses $A^{-1} \in \Mat(3,\ZZ)$ mit
  \[
    \det(A^{-1}) \eq \det(A^{-1}) \cdot 1 \eq
    \det(A^{-1}) \cdot \det(A) \eq 1
  \]
  Also ist $A^{-1} \in \SL(3,\ZZ)$.
\end{proof}


\subsection*{c)}
Wir bestimmen zunächst die Determinante von $M$
mit der Regel von Sarrus:
\[
  \det(M) = (b \cdot 3 \cdot c) + (0 \cdot 1 \cdot 2)
  + (1 \cdot 0 \cdot 2) - (2 \cdot 3 \cdot 1) - (2 \cdot 1
  \cdot b) - (c \cdot 0 \cdot 0)
  = 3bc - 6 - 2b = 1.
\]
Also ist
$
  3bc - 6 - 2b = 1$,
  daraus folgt
  $b \cdot (3c-2) = 7.
$
Daher kann $b$ nur Teiler von $7$ sein.
Wir probieren systematisch:
\begin{align*}
  &\text{für $b = 1$:}   && 3c-2 = 7 &&\imp\qquad c = 3 \\
  &\text{für $b = -1$:}   && 3c-2 = -7 &&\imp\qquad c = -\frac{5}{3} \notin \ZZ \\
  &\text{für $b = 7$:}   && 3c-2 = 1 &&\imp\qquad c = 1 \\
  &\text{für $b = -7$:}   && 3c-2 = -1 &&\imp\qquad c = \frac{1}{3} \notin \ZZ
\end{align*}
Also ist $(b,c) \in \{(1,3),(7,1)\}$.



\newpage

\section*{Aufgabe 2}

\subsection*{a)}

Die Basen $B$ und $B_0$ sind gleich orientiert genau
dann, wenn $\det(B) > 0$

\begin{proof}
Die Basen sind nach Definition genau dann gleich orientiert,
wenn $\det(T^B_{B_0}) > 0$. Wie in \textit{Bemerkung 5.109}
gezeigt wurde, ''\textit{erhalten wir $T^B_{B_0}$
einfach durch Nebeneinanderschreiben der Spaltenvektoren in B}'',
also $T^B_{B_0} = B$.
\end{proof}

\subsection*{b)}
Wir betrachten die Basen
$B' := (b_1 , b_2 , b_3)$
und
$B'' := (b_5 , b_4 , b_1)$
mit
\[
b_1 := \begin{pmatrix} 1 \\ 2 \\ 3 \end{pmatrix}, \quad
b_2 := \begin{pmatrix} 1 \\ 1 \\ 0 \end{pmatrix}, \quad
b_3 := \begin{pmatrix} 0 \\ 1 \\ 1 \end{pmatrix}, \quad
b_4 := \begin{pmatrix} 2 \\ 4 \\ 4 \end{pmatrix}, \quad
b_5 := \begin{pmatrix} 0 \\ 2 \\ 4 \end{pmatrix} \in \mathbb{R}^3.
\]
Wir bestimmen nun die Determinante der Basen
mit der Regel von Sarrus:
\[
\det(B')
=
(1 \cdot 1 \cdot 1) + (1 \cdot 1 \cdot 3)
  + (0 \cdot 2 \cdot 3) - (3 \cdot 1 \cdot 0) - (0 \cdot 1
  \cdot 1) - (1 \cdot 2 \cdot 1)
= 2,
\]
\[
\det(B'')
=
(0 \cdot 4 \cdot 3) + (2 \cdot 2 \cdot 4)
  + (1 \cdot 2 \cdot 4) - (4 \cdot 4 \cdot 1) - (4 \cdot 2
  \cdot 0) - (3 \cdot 2 \cdot 2)
= -4.
\]
Da $\sign(\det(B')) \neq \sign(\det(B''))$,
sind $B'$ und $B''$ verschieden orientiert.


\subsection*{c)}
Sei $\sigma \in S_n$.
Dann sind die Basen $B := (b_1, \ldots, b_n)$
und $B^{(\sigma)} := (b_{\sigma(1)}, \ldots, b_{\sigma(n)})$
genau dann gleich orientiert wenn 
$\sgn(\sigma) = 1$.

\begin{proof}
 Es gilt
 $B^{(\sigma)} = T^B_{B^{(\sigma)}} \cdot B$,
 Wobei sich
 $T^B_{B^{(\sigma)}} = (t_{ij})$
 folgendermaßen ergibt:
 \[
  b_i \eq 0 \cdot b^{(\sigma)}_1
  + \ldots + 1 \cdot b^{(\sigma)}_{\sigma(i)}
  + \ldots + 0 \cdot b^{(\sigma)}_n
  \qquad\text{also}\qquad
  t_{ij} = \begin{cases*}
    1 & \text{falls $j = \sigma(i)$} \\
    0 & \text{sonst}
   \end{cases*}.
 \]
Wir bestimmen die Determinante mit der Leibnizformel:
 $\det(T^B_{B^{(\sigma)}})
 =
 \sum_{\varsigma \in S_n} \sgn(\varsigma) \cdot t_{1,\varsigma(1)}
 \cdot \ldots \cdot t_{n,\varsigma(n)}
 $,
 wobei für alle Summanden gilt
 \[
  \sgn(\varsigma) \cdot t_{1,\varsigma(1)}
  \cdot \ldots \cdot t_{n,\varsigma(n)} \neq 0
 \qquad\imp\qquad
 \forall i \in \{1, \ldots, n\} : t_{i, \varsigma(i)} \neq 0
 \qquad\imp\qquad
 \varsigma = \sigma.
 \]
 Also ist
 \[
 \det(T^B_{B^{(\sigma)}})
 \eq
 \sgn(\sigma) \cdot t_{1,\sigma(1)}
 \cdot \ldots \cdot t_{n,\sigma(n)}
 \eq 
 \sgn(\sigma) \cdot 1
 \cdot \ldots \cdot 1
 \eq
 \sgn(\sigma).\]
 Das bedeutet:
 \[
 \text{$B$ und $B^{(\sigma)}$ sind gleich orientiert}
 \qquad\equ\qquad
 \det(T^B_{B^{(\sigma)}}) > 0
 \qquad\equ\qquad
 \sgn(\sigma) = 1.
 \]
\end{proof}






\end{document}
