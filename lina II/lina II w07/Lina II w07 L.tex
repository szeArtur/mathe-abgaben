\documentclass{article}
\usepackage{graphicx, amsmath, amsthm, amssymb, mathtools, enumerate, bbm}
\usepackage{stmaryrd}
\usepackage[a4paper, total={6in, 8in}]{geometry}
\usepackage[T1]{fontenc}
\usepackage[ngerman]{babel}


\title{Lineare Algebra II (LA) Übungsblatt 7}
\author{Erik Achilles, Alexandra Dittmar, Artur Szeczinowski}
\date{Mai 2025}




\setlength{\parindent}{0pt}

\newcommand{\NN}{\mathbb{N}}
\newcommand{\ZZ}{\mathbb{Z}}
\newcommand{\QQ}{\mathbb{Q}}
\newcommand{\RR}{\mathbb{R}}
\newcommand{\FF}{\mathbb{F}}
\newcommand{\CC}{\mathbb{C}}

\newcommand{\imp}{\mathbb{\Rightarrow}}
\newcommand{\equ}{\mathbb{\Leftrightarrow}}
\newcommand{\eq}{\mathbb{\quad = \quad}}

\newcommand{\limn}{\lim_{n \rightarrow \infty}}
\newcommand{\toInf}[1]{\overset{#1 \rightarrow \infty}{\longrightarrow}}

\newcommand{\movs}[2]{\overset{\text{\tiny $#1$}}{\quad #2 \quad}}
\newcommand{\tovs}[2]{\overset{\text{\tiny (#1)}}{\quad #2 \quad}}
\newcommand{\vect}[1]{\begin{pmatrix*}[c] #1 \end{pmatrix*}}
\newcommand{\sect}[1]{\begin{psmallmatrix*}[c] #1 \end{psmallmatrix*}}
\newcommand{\legs}[2]{\left(\begin{array}{#1}#2\end{array}\right)}

\DeclareMathOperator{\RRe}{Re}
\DeclareMathOperator{\IIm}{Im}
\DeclareMathOperator{\Mat}{Mat}
\DeclareMathOperator{\im}{im}
\DeclareMathOperator{\rg}{rg}
\DeclareMathOperator{\LH}{L}
\DeclareMathOperator{\Bas}{Bas}
\DeclareMathOperator{\Kern}{ker}
\DeclareMathOperator{\Abb}{Abb}
\DeclareMathOperator{\Fin}{Fin}
\DeclareMathOperator{\Konv}{Konv}
\DeclareMathOperator{\Poly}{Poly}
\DeclareMathOperator{\sign}{sign}
\DeclareMathOperator{\sgn}{sgn}
\DeclareMathOperator{\GL}{GL}
\DeclareMathOperator{\SL}{SL}
\DeclareMathOperator{\vol}{vol}

%% https://texblog.net/latex-archive/maths/amsmath-matrix/
\makeatletter
\renewcommand*\env@matrix[1][*\c@MaxMatrixCols c]{%
  \hskip -\arraycolsep
  \let\@ifnextchar\new@ifnextchar
  \array{#1}}
\makeatother




\begin{document}
%\maketitle
%\newpage

\section*{Aufgabe 2}
\subsection*{a)}
Wir berechnen für $\alpha, \beta \in \RR$ das Produkt:

\[
  \sum_{j=0}^{\infty} \frac{\alpha^j}{j!} X^j \cdot
  \sum_{j=0}^{\infty} \frac{\beta^j}{j!} X^j
\]

Wenn $a_j = \frac{\alpha^j}{j!}$ und $b_j = \frac{\beta^j}{j!}$, dann ist
das Produkt

\[
  f \cdot g = \sum_{k=0}^{\infty} c_k X^k
\]

mit den Koeffizienten

\[
  \begin{aligned}
    c_k & = \sum_{i=0}^{k} a_i b_{k-i} = \sum_{i=0}^{k} \frac{\alpha^i}{i!} \cdot \frac{\beta^{k-i}}{(k-i)!} = \sum_{i=0}^{k} \frac{1}{i! (k-i)!} \alpha^i \beta^{k-i} \\
        & = \frac{1}{k!} \sum_{i=0}^{k} \binom{k}{i} \alpha^i \beta^{k-i} = \frac{(\alpha + \beta)^k}{k!}.
  \end{aligned}
\]

Also ist

\[
  \sum_{j=0}^{\infty} \frac{\alpha^j}{j!} X^j \cdot \sum_{j=0}^{\infty} \frac{\beta^j}{j!} X^j
  = \sum_{k=0}^{\infty} \frac{(\alpha + \beta)^k}{k!} X^k = \exp((\alpha + \beta) X).
\]

Wir berechnen das Produkt:

\[
  (1 - 2X) \cdot \sum_{j=0}^{\infty} 2^j X^j.
\]

Wir nutzen dazu die Formel für die geometrische Reihe:

\[
  \sum_{j=0}^{\infty} 2^j X^j = \frac{1}{1 - 2X}.
\]

Also ist

\[
  (1 - 2X) \cdot \sum_{j=0}^{\infty} 2^j X^j = (1 - 2X) \cdot \frac{1}{1 - 2X} = 1.
\]


\newpage
\subsection*{b)}


Für $f,g,h \in K[X]$ ist $(f \cdot g) \cdot h = f \cdot (g \cdot h)$.

\begin{proof}
  Sei $f = \sum_{i=0}^N a_i X^i$, $g = \sum_{i=0}^M b_i X^i$ und $h = \sum_{i=0}^L c_i X^i$.
  Dann ist
  \[
    \begin{aligned}
      (f \cdot g) \cdot h & = \left( \sum_{i=0}^N a_i X^i \cdot \sum_{i=0}^M b_i X^i \right) \cdot \sum_{i=0}^L c_i X^i \\
                          & = \left(\sum_{k} \left(\sum_{i+j=k} a_i b_j\right) X^k\right) \cdot \sum_{i=0}^L c_i X^i    \\
                          & = \sum_s \left(\sum_{k+l=s} c_l \left(\sum_{i+j=k} a_i b_j\right)\right) X^s                \\
                          & = \sum_s \left(\sum_{i+j+k=s} a_i b_j c_k\right) X^s                                        \\
                          & = \sum_s \left(\sum_{i+k=s} a_i \left(\sum_{j+l=k} b_j c_l\right)\right) X^s                \\
                          & = \sum_{s} a_s X^s \cdot \left(\sum_{k} \left( \sum_{j+l=k} b_j c_l\right) X^k\right)       \\
                          & = \sum_{s} a_s X^s \cdot \left(\sum_{k} b_k X^k \cdot \sum_{l} c_l X^l\right)               \\
                          & = f \cdot (g \cdot h).
    \end{aligned}
  \]
  Also ist $(f \cdot g) \cdot h = f \cdot (g \cdot h)$.
\end{proof}


\newpage

Sei $K$ ein Körper. Für $f \in K[X]$ mit $f = \sum_{i=0}^N a_i X^i$
definieren wir die Abbildung

\[
  \tilde{f} : K \to K, x \mapsto \sum_{i=0}^{n} a_i x^i.
\]

Für $f,g \in K[X]$ gilt $\widetilde{f \cdot g} = \tilde{f} \cdot \tilde{g} $.
\begin{proof}
  Sei $f = \sum_{i=0}^N a_i X^i$ und $g = \sum_{i=0}^M b_i X^i$.
  Zudem sei $x \in K$.
  Dann ist

  \[
    \begin{aligned}
      (\widetilde{f \cdot g})(x)
       & = \widetilde{\left( \sum_k \left( \sum_{i+j=k} a_i b_j \right) X^k \right)}(x) \\
       & = \sum_k \left( \sum_{i+j=k} a_i b_j \right) x^k                               \\
       & = \sum_i a_i x^i \cdot \sum_j b_j x^j                                          \\
       & = \left(\sum_i a_i x^i \right)(x) \cdot \left(\sum_j b_j X^j\right)(x)         \\
       & = \widetilde{f}(x) \cdot \widetilde{g}(x)                                      \\
       & = (\tilde{f} \cdot \tilde{g})(x).
    \end{aligned}
  \]
\end{proof}
\end{document}
