\documentclass{article}
\usepackage{graphicx, amsmath, amsthm, amssymb, mathtools, enumerate, bbm}
\usepackage[a4paper, total={6in, 8in}]{geometry}
\usepackage[T1]{fontenc}
\usepackage[ngerman]{babel}


\title{Lineare Algebra II (LA) Übungsblatt 7}
\author{Erik Achilles, Alexandra Dittmar, Artur Szeczinowski}
\date{Mai 2025}




\setlength{\parindent}{0pt}

\newcommand{\NN}{\mathbb{N}}
\newcommand{\ZZ}{\mathbb{Z}}
\newcommand{\QQ}{\mathbb{Q}}
\newcommand{\RR}{\mathbb{R}}
\newcommand{\FF}{\mathbb{F}}
\newcommand{\CC}{\mathbb{C}}

\newcommand{\imp}{\mathbb{\Rightarrow}}
\newcommand{\equ}{\mathbb{\Leftrightarrow}}
\newcommand{\eq}{\mathbb{\quad = \quad}}

\newcommand{\limn}{\lim_{n \rightarrow \infty}}
\newcommand{\toInf}[1]{\overset{#1 \rightarrow \infty}{\longrightarrow}}

\newcommand{\movs}[2]{\overset{\text{\tiny $#1$}}{\quad #2 \quad}}
\newcommand{\tovs}[2]{\overset{\text{\tiny (#1)}}{\quad #2 \quad}}
\newcommand{\vect}[1]{\begin{pmatrix*}[c] #1 \end{pmatrix*}}
\newcommand{\sect}[1]{\begin{psmallmatrix*}[c] #1 \end{psmallmatrix*}}
\newcommand{\legs}[2]{\left(\begin{array}{#1}#2\end{array}\right)}

\DeclareMathOperator{\RRe}{Re}
\DeclareMathOperator{\IIm}{Im}
\DeclareMathOperator{\Mat}{Mat}
\DeclareMathOperator{\im}{im}
\DeclareMathOperator{\rg}{rg}
\DeclareMathOperator{\LH}{L}
\DeclareMathOperator{\Bas}{Bas}
\DeclareMathOperator{\Kern}{ker}
\DeclareMathOperator{\Abb}{Abb}
\DeclareMathOperator{\Fin}{Fin}
\DeclareMathOperator{\Konv}{Konv}
\DeclareMathOperator{\Poly}{Poly}
\DeclareMathOperator{\sign}{sign}
\DeclareMathOperator{\sgn}{sgn}
\DeclareMathOperator{\GL}{GL}
\DeclareMathOperator{\SL}{SL}
\DeclareMathOperator{\vol}{vol}

%% https://texblog.net/latex-archive/maths/amsmath-matrix/
\makeatletter
\renewcommand*\env@matrix[1][*\c@MaxMatrixCols c]{%
  \hskip -\arraycolsep
  \let\@ifnextchar\new@ifnextchar
  \array{#1}}
\makeatother




\begin{document}
%\maketitle
%\newpage

\section*{Aufgabe 1}
\subsection*{a)}

\[
  \tilde{f} := K[X]
\]


Geometrische Reihe:

\[
  \text{Für } \frac{1}{1 - x} = \sum_{n=0}^{\infty} x^n \quad \text{für } |x| < 1
\]

Wir erkennen, dass dies die Taylorreihe der Exponentialfunktion ist:

\[
  \exp(\lambda X) = \sum_{n=0}^\infty \frac{(\lambda X)^n}{n!} = \exp(\lambda) \cdot \exp(X)
\]

Daraus folgt dann:

\[
  \exp(\lambda X) \cdot \exp(\mu X) = \exp((\lambda + \mu) X)
\]

Der binomische Lehrsatz sagt nun:

\[
  (x + y)^n = \sum_{k=0}^n \binom{n}{k} x^k y^{n-k}
\]

Somit ist das Produkt:

\[
  A(X) \cdot B(X) = e^B
\]

Für $1 - 2x$ gilt:

Hier erkennen wir die geometrische Reihe mit:

\[
  f(X) = \frac{1}{1 - 2x}
\]

Dies multiplizieren wir nun mit $(1 - 2x)$:

\[
  \frac{1}{1 - 2x} \cdot (1 - 2x) = 1
\]

Wir erhalten also als Produkt $R(X)$ die Potenzreihe 1.


\subsection*{b)}
Zeigen Sie für $F := K(X)$:

\[
  (fg) \cdot h = f \cdot (g \cdot h) \quad \text{sowie} \quad F = F_{ij}
\]

\[
  (f \cdot g) \cdot h = f \cdot (g \cdot h)
\]

Für $f_i, g_i, h_i$ Objekte und $p_e$:

\[
  f \cdot g = d \quad \text{mit} \quad (f \cdot g) \cdot h
\]

\[
  V_s = d_{ij} b_j
\]

\[
  f(g \cdot h): V = s_i \cdot e_r \quad \text{mit} \quad e_r: r \to b_j e_r
\]

\[
  = e_s i r b_j e_r = d_i b_j e_r = V_s
\]

Für alle $s$ folgt also:

\[
  (f \cdot g) \cdot h = f \cdot (g \cdot h)
\]

Also:

\[
  i + j + l = s
\]


\end{document}
