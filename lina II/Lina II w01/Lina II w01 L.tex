\documentclass{article}
\usepackage{graphicx, amsmath, amsthm, amssymb, mathtools, enumerate}
%\usepackage{virus}
\usepackage[a4paper, total={6in, 8in}]{geometry}
\usepackage[T1]{fontenc}
\usepackage[ngerman]{babel}


\title{Lineare Algebra II (LA) Übungsblatt 1}
\author{Erik Achilles, Alexandra Dittmar, Artur Szeczinowski}
\date{April 2025}




\setlength{\parindent}{0pt}


\newcommand{\NN}{\mathbb{N}}
\newcommand{\ZZ}{\mathbb{Z}}
\newcommand{\QQ}{\mathbb{Q}}
\newcommand{\RR}{\mathbb{R}}
\newcommand{\FF}{\mathbb{F}}
\newcommand{\CC}{\mathbb{C}}

\newcommand{\imp}{\mathbb{\Rightarrow}}
\newcommand{\equ}{\mathbb{\Leftrightarrow}}
\newcommand{\eq}{\mathbb{\quad = \quad}}

\DeclareMathOperator{\RRe}{Re}
\DeclareMathOperator{\IIm}{Im}
\DeclareMathOperator{\Mat}{Mat}
\DeclareMathOperator{\im}{im}
\DeclareMathOperator{\rg}{rg}
\DeclareMathOperator{\LH}{L}
\DeclareMathOperator{\Bas}{Bas}
\DeclareMathOperator{\Kern}{ker}
\DeclareMathOperator{\Abb}{Abb}
\DeclareMathOperator{\Fin}{Fin}
\DeclareMathOperator{\Konv}{Konv}
\DeclareMathOperator{\Poly}{Poly}

\newcommand{\limn}{\lim_{n \rightarrow \infty}}
\newcommand{\toInf}[1]{\overset{#1 \rightarrow \infty}{\longrightarrow}}

\newcommand{\movs}[2]{\overset{\text{\tiny $#1$}}{\quad #2 \quad}}
\newcommand{\tovs}[2]{\overset{\text{\tiny (#1)}}{\quad #2 \quad}}
\newcommand{\vect}[1]{\begin{pmatrix*}[c] #1 \end{pmatrix*}}
\newcommand{\sect}[1]{\begin{psmallmatrix*}[c] #1 \end{psmallmatrix*}}
\newcommand{\legs}[2]{\left(\begin{array}{#1}#2\end{array}\right)}


%% https://texblog.net/latex-archive/maths/amsmath-matrix/
\makeatletter
\renewcommand*\env@matrix[1][*\c@MaxMatrixCols c]{%
  \hskip -\arraycolsep
  \let\@ifnextchar\new@ifnextchar
  \array{#1}}
\makeatother




\begin{document}
\maketitle
\section*{Aufgabe 2}

\subsection*{a)}

Wir betrachten die Abbildung
$\varphi_M : \RR^3 \to \RR^3, x \mapsto M \cdot x$
für
\[
M :=
\legs{ccc}{
    -1 & 1 & 0  \\
    0 & -1 & -1 \\
    2 & 1 & 3 
}
\in \Mat(3,R).
\]
Insbesondere haben wir
$V = W = \RR^3$
und
$n := \dim(V) = 3$
Weiter gilt
$r := \dim(\im(\varphi)) = \rg(M) = 2$
und nach Dimensionsformel 
$\dim(\Kern(\varphi)) = 4 - \rg(M) = 2$.
Wir konstruieren nun schrittweise die gewünschten Basen:

1. Wähle Basis
$B' := (b_1,\ldots,b_r )$
von 
$\im(\varphi) und a_i \in V mit \varphi(a_i) = b_i (i=1,\ldots,r)$
Die ersten beiden Spalten von M sind linear unabhängig und bilden wegen
$r = \rg(M) = 2$ eine Basis
$B' = (b_1,b_2)$
vom Bild. Wir setzen also
\[
    b_1 := (-1,0,2)t
    b_2 := (1,-1,1)t
    a_1 := e_1
    a_2 := e_2.
\]

2. Wähle Basis
$A' := (a_{r +1},...,a_n)$
 aus $n -r$ Vektoren von
$\Kern(\varphi)$.
Hier gilt
$n -r = 2$
und wir wählen die Basis
$A' = (a_3,a_4)$ von 
$\Kern(\varphi) = \text{Lös}(M,0)$
mit
\[a_3 := (1,1,-1,0)^t
a_4 := (1,-1,0,1)t
.\]
3. Ergänze $B'$
zu einer Basis
$B = (b_1,...,b_r
,...,b_m)$ von $W$
Konkret wählen wir
$B := (b_1,b_2,b_3)$
mit
$b_3 := (0,1,0)^t$

4. Zeige:
$A := (a_1,...,a_r
,a_{r +1},...,a_n)$
ist linear unabhängig und wegen
$n = \dim(V )$
Basis von
$V$
Wir könnten direkt prüfen, dass $A$ Basis ist, gehen hier aber anders vor: Gelte
$\lambda_1 \cdot a_1 +\lambda_2 \cdot a_2 +\lambda_3 \cdot a_3 +\lambda_4 \cdot a_4 = 0$
Wir multiplizieren mit $M$ und erhalten wegen
$a_3,a_4 \in \Kern(\varphi_M )$:
\[
    0 = M \cdot(\lambda_1 \cdot a_1 +...+\lambda_4 \cdot a_4) = \lambda_1 \cdot b_1 +\lambda_2 \cdot b_2 +\lambda_3 \cdot_0+\lambda_4 \cdot_0 = \lambda_1 \cdot b_1 +\lambda_2 \cdot b_2.
\]

$B' = (b1,b2)$
 ist linear unabhängig, es folgt
 $ \lambda1 = \lambda2 = 0$.
 Da
 $A' = (a_3,a_4)$
 auch linear unabhängig ist, folgt: 
 $\lambda3 = \lambda4 = 0$.
 Also ist $A$ linear unabhängig und damit Basis.
5. Verifiziere, dass 
$M^A_B(\varphi)$
 die gewünschte Form hat
Wir wenden direkt Verfahren 16 an:
\begin{align*}
    \varphi(a_1) \eq M \cdot e_1 = b_1 = 1 \cdot b_1 +0 \cdot b_2 +0 \cdot b_3 \\
    \varphi(a_2) \eq M \cdot e_2 = b_2 = 0 \cdot b_1 +1 \cdot b_2 +0 \cdot b_3 \\
    \varphi(a_3) \eq \varphi(a_4) = 0 = 0 \cdot b_1 +0 \cdot b_2 +0 \cdot b_3
\end{align*}
Also erhalten wir eine darstellenden Matrix in der gewünschten Form:
\[
    M^A_B
    (\varphi_M ) =
    \legs{cccc}{
    1 & 0 & 0 & 0 \\
    0 & 1 & 0 & 0 \\
    0 & 0 & 0 & 0
    }
\]


\end{document}
