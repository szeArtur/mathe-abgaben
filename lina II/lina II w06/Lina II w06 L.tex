\documentclass{article}
\usepackage{graphicx, amsmath, amsthm, amssymb, mathtools, enumerate, bbm}
%\usepackage{virus}
\usepackage[a4paper, total={6in, 8in}]{geometry}
\usepackage[T1]{fontenc}
\usepackage[ngerman]{babel}


\title{Lineare Algebra II (LA) Übungsblatt 5}
\author{Erik Achilles, Alexandra Dittmar, Artur Szeczinowski}
\date{Mai 2025}




\setlength{\parindent}{0pt}


\newcommand{\NN}{\mathbb{N}}
\newcommand{\ZZ}{\mathbb{Z}}
\newcommand{\QQ}{\mathbb{Q}}
\newcommand{\RR}{\mathbb{R}}
\newcommand{\FF}{\mathbb{F}}
\newcommand{\CC}{\mathbb{C}}

\newcommand{\imp}{\mathbb{\Rightarrow}}
\newcommand{\equ}{\mathbb{\Leftrightarrow}}
\newcommand{\eq}{\mathbb{\quad = \quad}}

\DeclareMathOperator{\RRe}{Re}
\DeclareMathOperator{\IIm}{Im}
\DeclareMathOperator{\Mat}{Mat}
\DeclareMathOperator{\im}{im}
\DeclareMathOperator{\rg}{rg}
\DeclareMathOperator{\LH}{L}
\DeclareMathOperator{\Bas}{Bas}
\DeclareMathOperator{\Kern}{ker}
\DeclareMathOperator{\Abb}{Abb}
\DeclareMathOperator{\Fin}{Fin}
\DeclareMathOperator{\Konv}{Konv}
\DeclareMathOperator{\Poly}{Poly}
\DeclareMathOperator{\sign}{sign}
\DeclareMathOperator{\sgn}{sgn}
\DeclareMathOperator{\GL}{GL}
\DeclareMathOperator{\SL}{SL}
\DeclareMathOperator{\vol}{vol}

\newcommand{\limn}{\lim_{n \rightarrow \infty}}
\newcommand{\toInf}[1]{\overset{#1 \rightarrow \infty}{\longrightarrow}}

\newcommand{\movs}[2]{\overset{\text{\tiny $#1$}}{\quad #2 \quad}}
\newcommand{\tovs}[2]{\overset{\text{\tiny (#1)}}{\quad #2 \quad}}
\newcommand{\vect}[1]{\begin{pmatrix*}[c] #1 \end{pmatrix*}}
\newcommand{\sect}[1]{\begin{psmallmatrix*}[c] #1 \end{psmallmatrix*}}
\newcommand{\legs}[2]{\left(\begin{array}{#1}#2\end{array}\right)}


%% https://texblog.net/latex-archive/maths/amsmath-matrix/
\makeatletter
\renewcommand*\env@matrix[1][*\c@MaxMatrixCols c]{%
  \hskip -\arraycolsep
  \let\@ifnextchar\new@ifnextchar
  \array{#1}}
\makeatother




\begin{document}
%\maketitle
%\newpage
\section*{Aufgabe 2}

\subsection*{a)}

Für ein festes $n \in \NN$ und $s \in (0, \infty)$
sei die Abbildung
$$
F_s: B_{R}(0) \to \RR^n , x \mapsto (s \cdot \mathbbm{1}_n) \cdot x.
$$
Für jedes $x \in B_R(0)$ gilt $\Vert x \Vert \leq R$,
also folgt für $F_s(x) = s \cdot \mathbbm{1}_n \cdot x = s \cdot x$,
dass $\Vert s \cdot x \Vert = s \cdot \Vert x \Vert \leq s \cdot R$,
und somit $F_s(x) \in B_{s\cdot R}(0)$.
Da $\det(s \cdot \mathbbm{1}_n) = s^n \neq 0$,
ist $F_s$ bijektiv.
Also ist $F_s(B_{R}(0)) = B_{s \cdot R}(0)$ und es gilt:
$$
\vol(B_{s \cdot R}(0)) \eq
\vol((s \cdot \mathbbm{1}_n) \cdot B_{R}(0)) \eq
|\det(s \cdot \mathbbm{1}_n)| \cdot \vol(B_R(0)) \eq 
s^n \cdot \vol(B_R(0))
$$
Sei nun $s = \frac{19}{20}$.
Wir berechnen
\begin{align*}
  n=3: \qquad & \frac{\vol(B_{s\cdot R}(0))}{\vol(B_R(0))}
  = s^n = \left(\frac{19}{20}\right)^3 \;\approx\; 0.86 \\
  n=10: \qquad & \frac{\vol(B_{s\cdot R}(0))}{\vol(B_R(0))}
  = s^n = \left(\frac{19}{20}\right)^{10} \;\approx\; 0.60 \\
  n=25: \qquad & \frac{\vol(B_{s\cdot R}(0))}{\vol(B_R(0))}
  = s^n = \left(\frac{19}{20}\right)^{25} \;\approx\; 0.28
\end{align*}

\subsection*{b)}
Wir bestimmen das kleinste $n \in \NN$
mit einem Schalenanteil von über $\frac{999999}{1000000}$:
\begin{align*}
  &1 -\frac{\vol(B_{s\cdot R}(0))}{\vol(B_R(0))} = 1- \left(\frac{19}{20}\right)^n
  \geq \frac{999999}{1000000} \\
  \equ\qquad &\left(\frac{19}{20}\right)^n \leq \frac{1}{1000000} \\
  \equ\qquad &n \geq \log_{\frac{19}{20}}\left(\frac{1}{1000000}\right) \approx 269.34 \\
  \imp \qquad & n = \lceil 269.34 \rceil = 270.
\end{align*}

\end{document}
