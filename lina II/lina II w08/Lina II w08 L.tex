\documentclass{article}
\usepackage{graphicx, amsmath, amsthm, amssymb, mathtools, enumerate, bbm}
\usepackage{stmaryrd}
\usepackage[a4paper, total={6in, 8in}]{geometry}
\usepackage[T1]{fontenc}
\usepackage[ngerman]{babel}


\title{Lineare Algebra II (LA) Übungsblatt 7}
\author{Erik Achilles, Alexandra Dittmar, Artur Szeczinowski}
\date{Mai 2025}




\setlength{\parindent}{0pt}

\newcommand{\NN}{\mathbb{N}}
\newcommand{\ZZ}{\mathbb{Z}}
\newcommand{\QQ}{\mathbb{Q}}
\newcommand{\RR}{\mathbb{R}}
\newcommand{\FF}{\mathbb{F}}
\newcommand{\CC}{\mathbb{C}}

\newcommand{\imp}{\mathbb{\Rightarrow}}
\newcommand{\equ}{\mathbb{\Leftrightarrow}}
\newcommand{\eq}{\mathbb{\quad = \quad}}

\newcommand{\limn}{\lim_{n \rightarrow \infty}}
\newcommand{\toInf}[1]{\overset{#1 \rightarrow \infty}{\longrightarrow}}

\newcommand{\movs}[2]{\overset{\text{\tiny $#1$}}{\quad #2 \quad}}
\newcommand{\tovs}[2]{\overset{\text{\tiny (#1)}}{\quad #2 \quad}}
\newcommand{\vect}[1]{\begin{pmatrix*}[c] #1 \end{pmatrix*}}
\newcommand{\sect}[1]{\begin{psmallmatrix*}[c] #1 \end{psmallmatrix*}}
\newcommand{\legs}[2]{\left(\begin{array}{#1}#2\end{array}\right)}

\DeclareMathOperator{\RRe}{Re}
\DeclareMathOperator{\IIm}{Im}
\DeclareMathOperator{\Mat}{Mat}
\DeclareMathOperator{\im}{im}
\DeclareMathOperator{\rg}{rg}
\DeclareMathOperator{\LH}{L}
\DeclareMathOperator{\Bas}{Bas}
\DeclareMathOperator{\Kern}{ker}
\DeclareMathOperator{\Abb}{Abb}
\DeclareMathOperator{\Fin}{Fin}
\DeclareMathOperator{\Konv}{Konv}
\DeclareMathOperator{\Poly}{Poly}
\DeclareMathOperator{\sign}{sign}
\DeclareMathOperator{\sgn}{sgn}
\DeclareMathOperator{\GL}{GL}
\DeclareMathOperator{\SL}{SL}
\DeclareMathOperator{\vol}{vol}

%% https://texblog.net/latex-archive/maths/amsmath-matrix/
\makeatletter
\renewcommand*\env@matrix[1][*\c@MaxMatrixCols c]{%
  \hskip -\arraycolsep
  \let\@ifnextchar\new@ifnextchar
  \array{#1}}
\makeatother




\begin{document}
%\maketitle
%\newpage

\section*{Aufgabe 1}
\subsection*{a)}

Sei $f = \sum_{k=0}^{N}a_kX^k\in \CC[X]$ mit
$a_k \in \RR$ für $k \leq N \in \NN$ und
$\lambda \in \CC$ eine Nullstelle, so ist die
komplex konjugierte Zahl $\overline{\lambda} \in \CC$
ebenfalls eine Nullstelle

\begin{proof}
  Gelte $\tilde{f}(\lambda) = 0$.
  Wir nutzen im folgenden, dass für $z\in\CC,a\in\RR$ gilt

  \[
  a \cdot \overline{z} = \overline{a \cdot z},
  \qquad\qquad
  \overline{z} + \overline{z} = \overline{z + z},
  \qquad\qquad
  \overline{z} \cdot \overline{z} = \overline{z \cdot z}.
  \]

  Also ist $\overline{\lambda} \in \CC$ eine Nullstelle von $f$,
  denn:

  \[
  \tilde{f}(\overline{\lambda})
  \eq
  \sum_{k=0}^{N}a_k \overline{\lambda}^k
  \eq
  \overline{\sum_{k=0}^{N}a_k \lambda^k}
  \eq
  \overline{\tilde{f}(\lambda)}
  \eq
  \overline{0}
  \eq
  0.
  \]
\end{proof}

\end{document}
