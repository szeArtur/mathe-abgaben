\documentclass{article}
\usepackage{graphicx, amsmath, amsthm, amssymb, mathtools, enumerate, bbm}
\usepackage{stmaryrd}
\usepackage[a4paper, total={6in, 8in}]{geometry}
\usepackage[T1]{fontenc}
\usepackage[ngerman]{babel}


\title{Lineare Algebra II (LA) Übungsblatt 7}
\author{Erik Achilles, Alexandra Dittmar, Artur Szeczinowski}
\date{Mai 2025}




\setlength{\parindent}{0pt}

\newcommand{\NN}{\mathbb{N}}
\newcommand{\ZZ}{\mathbb{Z}}
\newcommand{\QQ}{\mathbb{Q}}
\newcommand{\RR}{\mathbb{R}}
\newcommand{\FF}{\mathbb{F}}
\newcommand{\CC}{\mathbb{C}}

\newcommand{\imp}{\mathbb{\Rightarrow}}
\newcommand{\equ}{\mathbb{\Leftrightarrow}}
\newcommand{\eq}{\mathbb{\quad = \quad}}

\newcommand{\limn}{\lim_{n \rightarrow \infty}}
\newcommand{\toInf}[1]{\overset{#1 \rightarrow \infty}{\longrightarrow}}

\newcommand{\movs}[2]{\overset{\text{\tiny $#1$}}{\quad #2 \quad}}
\newcommand{\tovs}[2]{\overset{\text{\tiny (#1)}}{\quad #2 \quad}}
\newcommand{\vect}[1]{\begin{pmatrix*}[c] #1 \end{pmatrix*}}
\newcommand{\sect}[1]{\begin{psmallmatrix*}[c] #1 \end{psmallmatrix*}}
\newcommand{\legs}[2]{\left(\begin{array}{#1}#2\end{array}\right)}

\DeclareMathOperator{\RRe}{Re}
\DeclareMathOperator{\IIm}{Im}
\DeclareMathOperator{\Mat}{Mat}
\DeclareMathOperator{\im}{im}
\DeclareMathOperator{\rg}{rg}
\DeclareMathOperator{\LH}{L}
\DeclareMathOperator{\Bas}{Bas}
\DeclareMathOperator{\Kern}{ker}
\DeclareMathOperator{\Abb}{Abb}
\DeclareMathOperator{\Fin}{Fin}
\DeclareMathOperator{\Konv}{Konv}
\DeclareMathOperator{\Poly}{Poly}
\DeclareMathOperator{\sign}{sign}
\DeclareMathOperator{\sgn}{sgn}
\DeclareMathOperator{\GL}{GL}
\DeclareMathOperator{\SL}{SL}
\DeclareMathOperator{\vol}{vol}

%% https://texblog.net/latex-archive/maths/amsmath-matrix/
\makeatletter
\renewcommand*\env@matrix[1][*\c@MaxMatrixCols c]{%
  \hskip -\arraycolsep
  \let\@ifnextchar\new@ifnextchar
  \array{#1}}
\makeatother




\begin{document}
%\maketitle
%\newpage

\section*{Aufgabe 1}
\subsection*{a)}

Sei $f = \sum_{k=0}^{N}a_kX^k\in \CC[X]$ mit
$a_k \in \RR$ für $k \leq N \in \NN$ und
$\lambda \in \CC$ eine Nullstelle von $f$, so ist die
komplex konjugierte Zahl $\overline{\lambda} \in \CC$
ebenfalls eine Nullstelle.

\begin{proof}
  Gelte $\tilde{f}(\lambda) = 0$.
  Wir nutzen im Folgenden, dass für $z\in\CC,a\in\RR$ gilt

  \[
  a \cdot \overline{z} = \overline{a \cdot z},
  \qquad\qquad
  \overline{z} + \overline{z} = \overline{z + z},
  \qquad\qquad
  \overline{z} \cdot \overline{z} = \overline{z \cdot z}.
  \]

  Also ist $\overline{\lambda} \in \CC$ eine Nullstelle von $f$,
  denn:

  \[
  \tilde{f}(\overline{\lambda})
  \eq
  \sum_{k=0}^{N}a_k \overline{\lambda}^k
  \eq
  \overline{\sum_{k=0}^{N}a_k \lambda^k}
  \eq
  \overline{\tilde{f}(\lambda)}
  \eq
  \overline{0}
  \eq
  0.
  \]
\end{proof}

\subsection*{b)}

Wir betrachten die reelle $3 \times 3$-Matrix

\[
A := 
\begin{pmatrix}
1 & -2 & 0 \\
-2 & 5 & -4 \\
0 & 4 & -1
\end{pmatrix}
\quad \text{sowie} \quad
A - X \cdot \mathbbm{1}_3 := 
\begin{pmatrix}
1 - X & -2 & 0 \\
-2 & 5 - X & -4 \\
0 & 4 & -1 - X
\end{pmatrix}.
\]

Wir berechnen

\[
\begin{aligned}
  f_A
  &=
  \det(A - X \cdot \mathbbm{1}_3 ) \\
  &=
  (-X+1)(-X+5)(-X-1)-4(-4)(-X+1)-(-X-1)(-2)(-2)\\
  &=
  -X^3+5X^2-11X+15.
\end{aligned}
\]

Dieses Polynom $f_A \in \RR[X] \subset \CC[X]$
hat eine Nullstelle $\lambda_1 \in \RR$.

\begin{proof}
  Das Polynom $f_A$ hat den Grad $\deg(f_A) = 3$
  und zerfällt somit in $3$ Linearfaktoren:

  \[
  f_A = a(X - \lambda_1)(X - \lambda_2)(X - \lambda_3) \qquad
  \text{für geeignete $a,\lambda_1,\lambda_2,\lambda_3 \in \CC$,}
  \]

  wobei $\lambda_1,\lambda_2,\lambda_3$
  alle Nullstellen von $f_A$ sind.
  Da alle Koeffizienten von $f_A$ reell sind, gibt es, wie in
  \textit{a)} gezeigt, zu jedem $\lambda_n$ eine
  zweite Nullstelle $\overline{\lambda_n}$.
  Da $f_A$ eine ungerade Anzahl an Nullstellen hat,
  muss für mindestens ein $\lambda_n$
  gelten $\lambda_n = \overline{\lambda_n}$.
  Dies ist äquivalent zu $\lambda_n \in \RR$.
\end{proof}

\newpage

\subsection*{c)}

Sei $\lambda_1 = \frac{p}{q} \in \QQ$ eine Nullstelle
von $f_A \in \QQ[X] \subset \CC[X]$, wobei $p,q \in \ZZ$
teilerfremd sind, 
dann gilt $p|15$ und $q|(-1)$.
Also ist $\lambda_1 \in \{ -15, -5, -3, -1, 1, 3, 5, 15 \}$.
Wir probieren systematisch

\[
\begin{aligned}
  \tilde{f}(-15) &= 4680\\
  \tilde{f}(-5) &= 320\\
  \tilde{f}(-3) &= 120\\
  \tilde{f}(-1) &= 32\\
  \tilde{f}(1) &= 8\\
  \tilde{f}(3) &= 0\\
  \tilde{f}(5) &= -40\\
  \tilde{f}(15) &= -2400\\
\end{aligned}
\]

und erkennen $\lambda_1 = 3$. Sei also $g := X - 3$.
Wir führen die Polynomdivision von $f_A:g$ durch:

\begin{enumerate}
    \item[0.] Wir prüfen: $\deg(g) \leq \deg(f_A)$. Dies ist wahr, also gehe zu Schritt 1.
    
    \item[1.] Wir setzen: $q_1 := -X^2$ und 
    \[
    f_1 := f_A - q_1 g  = 2X^2 - 11X + 15.
    \]
    Wir prüfen: $\deg(g) \leq \deg(f_1)$. Dies ist wahr, also gehe zu Schritt 2.
    
    \item[2.] Wir setzen: $q_2 := 2X$ und 
    \[
    f_2 := f_1 - q_2 g = -5X + 15.
    \]
    Wir prüfen: $\deg(g) \leq \deg(f_2)$. Dies ist wahr, also gehe zu Schritt 3.
    
    \item[3.] Wir setzen: $q_3 := -5$ und 
    \[
    f_3 := f_2 - q_3 g =  0.
    \]
    Wir prüfen: $\deg(g) \leq \deg(f_3)$. Dies ist falsch, also bricht der Algorithmus ab.
\end{enumerate}

Wir setzen $q := q_1 + q_2 + q_3 = -X^2 + 2X - 5$.
Das Polynom $f_A$ zerfällt also wie folgt:

\[
f_A = (X - 3)\cdot(-X^2 + 2X - 5)
\]

Wir bestimmen nun die restlichen Nullstellen
des quadratischen Polynoms $-X^2 + 2X - 5$
mit der $p$-$q$-Formel und erhalten die Nullstellen
von $f_A$:
\[
\lambda_1 = 3, \lambda_{2/3} = 1 \pm 2i.
\]

\end{document}
