\documentclass{article}

\usepackage[utf8]{inputenc}
\usepackage{graphicx, amsmath, amsthm, amssymb, mathtools, enumerate, bbm, graphicx}
\usepackage{stmaryrd}
\usepackage[a4paper, total={6in, 8in}]{geometry}
\usepackage[T1]{fontenc}
\usepackage[ngerman]{babel}


\title{Lineare Algebra II (LA) Übungsblatt 13}
\author{Erik Achilles, Alexandra Dittmar, Artur Szeczinowski}
\date{Juli 2025}



\setlength{\parindent}{0pt}



\newcommand{\NN}{\mathbb{N}}
\newcommand{\ZZ}{\mathbb{Z}}
\newcommand{\QQ}{\mathbb{Q}}
\newcommand{\RR}{\mathbb{R}}
\newcommand{\FF}{\mathbb{F}}
\newcommand{\CC}{\mathbb{C}}

\newcommand{\imp}{\mathbb{\Rightarrow}}
\newcommand{\equ}{\mathbb{\Leftrightarrow}}
\newcommand{\eq}{\mathbb{\quad = \quad}}

\newcommand{\limto}[2]{\lim_{#1 \rightarrow #2}}
\newcommand{\toinf}[1]{\overset{#1 \rightarrow \infty}{\longrightarrow}}

\DeclareMathOperator{\RRe}{Re}
\DeclareMathOperator{\IIm}{Im}
\DeclareMathOperator{\Mat}{Mat}
\DeclareMathOperator{\id}{id}
\DeclareMathOperator{\im}{im}
\DeclareMathOperator{\rg}{rg}
\DeclareMathOperator{\LH}{L}
\DeclareMathOperator{\Bas}{Bas}
\DeclareMathOperator{\Kern}{ker}
\DeclareMathOperator{\Abb}{Abb}
\DeclareMathOperator{\Fin}{Fin}
\DeclareMathOperator{\Konv}{Konv}
\DeclareMathOperator{\Poly}{Poly}
\DeclareMathOperator{\sign}{sign}
\DeclareMathOperator{\sgn}{sgn}
\DeclareMathOperator{\GL}{GL}
\DeclareMathOperator{\SL}{SL}
\DeclareMathOperator{\vol}{vol}
\DeclareMathOperator{\End}{End}
\DeclareMathOperator{\eigenraum}{Eig}


\begin{document}


\section*{Aufgabe 1}

 \subsection*{a)}
 Sei
 $\langle \cdot, \cdot \rangle$
 das Standartskalarprodukt auf
 $\RR^2$.
 Wir definierten die Abbildung

 \[
 \langle \cdot, \cdot \rangle_A :
 \RR^2 \times \RR^2
 \to \RR,
 (v,w) \mapsto \langle v, Aw \rangle
 \qquad \text{mit} \qquad
 A :=
 \begin{pmatrix}
    2 & -1 \\ -1 & 2
 \end{pmatrix}
 \in \Mat(2,\RR).
 \]

 Die Abbildung
 $\langle \cdot, \cdot \rangle_A$
 definiert ein Skalarprodukt.

 \begin{proof}
    Zunächst bestimmen wir eine Eigenbasis
    $B := (b_1, b_2)$ wobei

    \[
    \begin{aligned}
        &b_1 := \begin{pmatrix}
            1 \\ 1
        \end{pmatrix}
        \qquad\text{denn}\qquad
        A \cdot \begin{pmatrix}
            1 \\ 1
        \end{pmatrix} = \begin{pmatrix}
            1 \\ 1
        \end{pmatrix},
        \\
        &b_2 := \begin{pmatrix}
            1 \\ -1
        \end{pmatrix}
        \qquad\text{denn}\qquad
        A \cdot \begin{pmatrix}
            1 \\ -1
        \end{pmatrix} = 3\begin{pmatrix}
            1 \\ -1
        \end{pmatrix}
    \end{aligned}.
    \]

    Für beliebige
    $v,w \in \RR^2$
    gibt es also passende
    $\lambda_1, \lambda_2, \mu_1, \mu_2$,
    sodass:

    \[
    v = \lambda_1 b_1 + \lambda_2 b_2
    \qquad\text{und}\qquad
    w = \mu_1 b_1 + \mu_2 b_2.
    \]
    
    Wir zeigen nun die Eigenschaften des Skalarproduktes.
    
    1.) Die Abbildung ist bilinear,
    denn das Standartskalarprodukt ist bilinear
    und Matrixmultiplikation ist linear.

    2.) Die Abbildung ist symmetrisch,
    denn es gilt

    \[
    \begin{aligned}
        \langle v, w \rangle_A
           &=
        \langle \lambda_1 b_1 + \lambda_2 b_2,
        A(\mu_1 b_1 + \mu_2 b_2) \rangle
        \\ &=
        \langle \lambda_1 b_1 + \lambda_2 b_2,
        A\mu_1 b_1 + A\mu_2 b_2 \rangle
        \\ &=
        \langle \lambda_1 b_1 + \lambda_2 b_2,
        1\mu_1 b_1 + 3\mu_2 b_2 \rangle
        \\ &=
        1\lambda_1\mu_1\langle  b_1,
         b_1 \rangle
        +
        1\lambda_2\mu_1\langle  b_2,
         b_1 \rangle
        +
        3\lambda_1\mu_2\langle  b_1,
         b_2 \rangle
        +
        3\lambda_2\mu_2\langle  b_2,
         b_2 \rangle
        \\ &=
        \langle \mu_1 b_1 + \mu_2 b_2,
        1\lambda_1 b_1 + 3\lambda_2 b_2 \rangle
        \\ &=
        \langle \mu_1 b_1 + \mu_2 b_2,
        A\lambda_1 b_1 + A\lambda_2 b_2 \rangle
        \\ &=
        \langle \mu_1 b_1 + \mu_2 b_2,
        A(\lambda_1 b_1 + \lambda_2 b_2) \rangle
        \\ &=
        \langle w, v \rangle_A.
    \end{aligned}
    \]

    3.) Die Abbildung ist positiv definit, denn

    \[
        \langle v, v \rangle_A
        =
        1\lambda_1\lambda_1\langle  b_1,
         b_1 \rangle
        +
        1\lambda_2\lambda_1\langle  b_2,
         b_1 \rangle
        +
        3\lambda_1\lambda_2\langle  b_1,
         b_2 \rangle
        +
        3\lambda_2\lambda_2\langle  b_2,
         b_2 \rangle
        \geq
        0 + 0 + 0 + 0
        = 0,
    \]

    wobei hier genutzt wurde, dass 
    $\langle b_1, b_2 \rangle = 0$
    und $\lambda_1^2, \lambda_2^2 \geq 0$.
    Zuletzt gilt

    \[
    \langle v, v \rangle_A = 0
    \quad\Rightarrow \quad
    \langle v, Av \rangle = 0
    \quad\Rightarrow \quad
    v = 0 \qedhere.
    \]
 \end{proof}

\newpage

 \subsection*{b)}
 Sei
 $s > 0$
 und 
 $v := \begin{pmatrix}
    s \\ 0
 \end{pmatrix},
 w := \begin{pmatrix}
    0 \\ s
 \end{pmatrix}
 $.
 Dann ist
 $Av = \begin{pmatrix}
    2s \\ -2
 \end{pmatrix}$
 und
 $Aw = \begin{pmatrix}
    -2 \\ 2s 
 \end{pmatrix}$.
 Wir berechnen

 \[
\begin{aligned}
     \sphericalangle_A (v,w)
     &=
     \arccos \left( \frac{\langle v , w \rangle_A}
     {\| v\|_A \cdot \|w\|_A} \right)
    \\ &=
     \arccos \left( \frac{\langle v , Aw \rangle}
     {\sqrt{\langle v,Av \rangle} \cdot \sqrt{\langle w,Aw \rangle}} \right)
    \\ &=
     \arccos \left( \frac{ - s^2 + 2s}
     {\sqrt{2s^2 - s} \cdot \sqrt{2s^2 - s}} \right)
    \\ &=
     \arccos \left( \frac{ - s^2 + 2s}
     {2s^2 - s} \right).
\end{aligned}
 \]

Wir berechnen

 \[
     \| v + w\|_A 
    =
    \sqrt{\langle v + w,A \cdot (v + w)\rangle}
    =
    \sqrt{\langle\begin{pmatrix}
        s \\ s
    \end{pmatrix}, A \begin{pmatrix} 
        s \\ s
    \end{pmatrix} \rangle}
    =
    \sqrt{\langle\begin{pmatrix}
        s \\ s
    \end{pmatrix}, \begin{pmatrix} 
        s \\ s
    \end{pmatrix} \rangle}
    =
    \sqrt{2s^2}
    =
    \sqrt{2} \cdot s
 \]

Wir berechnen

 \[
     \| v - w\|_A 
    =
    \sqrt{\langle v - w,A \cdot (v - w)\rangle}
    =
    \sqrt{\langle\begin{pmatrix}
        s \\ -s
    \end{pmatrix}, A \begin{pmatrix} 
        s \\ -s
    \end{pmatrix} \rangle}
    =
    \sqrt{\langle\begin{pmatrix}
        s \\ -s
    \end{pmatrix}, \begin{pmatrix} 
        3s \\ - 3s
    \end{pmatrix} \rangle}
    =
    \sqrt{3s^2 + 3s^2}
    =
    \sqrt{6} \cdot s
 \]


 \newpage

 \subsection*{c)}
 Wir visualisieren die Menge
 $\{ x \in \RR^2 | \| x \|_A^2 = 6\} \subset \RR^2$.
 Dazu berechnen wir
 
 \[
  \begin{aligned}
    \| x \|_A^2 = 6
    &\Rightarrow 
    \sqrt{\langle x,Ax \rangle}^2 = 6
    \\ &\Rightarrow
    | \langle x,Ax \rangle| = 6
    \\ &\Rightarrow
    \bigg|\biggl< \begin{pmatrix}
      x_1 \\x_2 
    \end{pmatrix}
    ,\begin{pmatrix} 
      2x_1 - x_2 \\ - x_1 + 2x_2
      \end{pmatrix}
      \biggr>\bigg| = 6
      \\ &\Rightarrow
      |x_1(2x_1 - x_2) + x_2( - x_1 + 2x_2) | = 6
      \\ &\Rightarrow
      |2x_1^2 - 2 x_1x_2 + 2x_2^2| = 6
      \\ &\Rightarrow
      |x_1^2 - x_1x_2 + x_2^2 | = 3.
  \end{aligned}
 \]

 Diese Gleichung lässt sich mit einem Grafikrechner plotten:

 \includegraphics*[width=0.9\textwidth]{circle.png}

\end{document}